
% Desenvolvido por: Prof. Dr. David Buzatto
% Adaptado por: Prof. Dr. Fernando Carvalho
%
% Baseado na documentação do abntex2 e nos modelos em
% Microsoft Word propostos pela Profa. Dra. Rosana F. L. Rodrigues
% e pela bibliotecária M.Sc. Maria Carolina Gonçalves do câmpus
% São João da Boa Vista do IFSP.
%
% Versão 1.5
% Data: 03/09/2018


\documentclass[
	% -- opções da classe memoir --
	12pt,				% tamanho da fonte
	openany,			% capítulos começam em pág ímpar openright (insere página vazia caso preciso)
	oneside,			% para impressão em verso e anverso (twoside). Oposto a oneside 
	a4paper,			% tamanho do papel. 
	normalfigtabnum,
	pnumromarab,
	% -- opções da classe abntex2 --
	chapter=Title,		% títulos de capítulos convertidos em letras maiúsculas
	section=Title,		% títulos de seções convertidos em letras maiúsculas
	%subsection=TITLE,	% títulos de subseções convertidos em letras maiúsculas
	%subsubsection=TITLE,% títulos de subsubseções convertidos em letras maiúsculas
	% -- opções do pacote babel --
	english,			% idioma adicional para hifenização
	french,				% idioma adicional para hifenização
	spanish,			% idioma adicional para hifenização
	brazil,				% o último idioma é o principal do documento
]{abntex2}

% \usepackage{blindtext, showframe}

% \usepackage[a4paper,showframe,
% top    = 2.75cm,
% bottom = 2.50cm,
% left   = 3.00cm,
% right  = 2.50cm]{geometry}


%%%%%%%%%%%%%%%%%%%%%%%%%%%%%%%%%%%%%%%%%%%%%%%%%%%%%%%%%%%%


\newenvironment{conditions}[1][onde:]
  {\noindent
	#1
	
	\indent
	\begin{tabular}[t]{>{$}l<{$} @{${} \> \> {}$} l}}
	{\end{tabular}\\[\belowdisplayskip]}


% --- Controla a geração de listas de siglas, símbolos e glossário
% \RequirePackage[style=long]{glossaries}
% ,long,numberlist


\usepackage[acronyms,symbols]{glossaries}
\setacronymstyle{long-short}

\makenoidxglossaries


%Definição das listas de glossários e parâmertos para compilação.
%-------------------------------------------------------
\newglossary{verbetes}{verbetes}{vrl}{verbetes}
\newglossary{siglas}{siglas}{sgl}{siglas}
\newglossary{simbolos}{simbolos}{sbl}{simbolos}


%Definição dos comandos para exibição das listas individuais.
%-----------------------------------------------------------
\newcommand{\listarsiglas}{\printglossary[type=siglas, title=\listadesiglasname, nonumberlist=true]}
\newcommand{\listarsimbolos}{\printglossary[type=simbolos, title=\listadesimbolosname, nonumberlist=true]}
\newcommand{\listarverbetes}{\printglossary[type=verbetes]}

%Definição dos Comandos para inclusão no texto das siglas.

\newcommand{\verbete}[3]{
	\newglossaryentry{glossary}{name=#1, description={#2}, plural={#3}, type=verbetes}
	\gls{#1}
}

\newcommand{\sigla}[2]{
	\newacronym[type=siglas]{#1}{#1}{#2}
	\gls{#1}
}

\newcommand{\simbolo}[3]{
	\newglossaryentry{#1}{type=simbolos, name=#3, text=#3, description=#2,symbol=#3, sort=def}
	\gls{#1}
}

%%%%%%%%%%%%%%%%%%%%%%%%%%%%%%%%%%%%%%%%%%%%%%%%%%%%%%%%%%%%

\RequirePackage{bookmark}   			



% ---------------------------------------------------------------------------------
%                                   PACOTES
% ---------------------------------------------------------------------------------


% ---
% Pacotes básicos 
% ---

\usepackage{cmap}
\usepackage{lmodern}

\usepackage{amsmath}
\usepackage{latexsym}
\usepackage{amsfonts}
\usepackage[normalem]{ulem}
\usepackage{array}
\usepackage{amssymb}
\usepackage{graphicx}

% \usepackage[backend=bibtex,
% style=authoryear,
% natbib=true,
% sorting=none,
% isbn=false,
% doi=false,
% url=false,

\usepackage{subfig}
\usepackage{wrapfig}
\usepackage{wasysym}
\usepackage{enumitem}
\usepackage{adjustbox}
\usepackage{ragged2e}
\usepackage[svgnames,table]{xcolor}
\usepackage{tikz}
\usetikzlibrary{intersections}
\usetikzlibrary{shapes,arrows,chains}
\tikzstyle{line}=[draw] % here
\usepackage{longtable}
\usepackage{changepage}
\usepackage{setspace}
\usepackage{hhline}
\usepackage{multicol}
\usepackage{tabto}
\usepackage{multirow}
\usepackage{makecell}
\usepackage{fancyhdr}



\usepackage{lmodern}			% Usa a fonte Latin Modern			
\usepackage[T1]{fontenc}		% Selecao de codigos de fonte.
\usepackage[utf8]{inputenc}		% Codificacao do documento (conversão automática dos acentos)
\usepackage{lastpage}			% Usado pela Ficha catalográfica
\usepackage{indentfirst}		% Indenta o primeiro parágrafo de cada seção.
\usepackage{xcolor,colortbl}	% Controle das cores
\usepackage{graphicx}			% Inclusão de gráficos
\usepackage{pgfplots}
\pgfplotsset{compat=1.17}
\usepackage{microtype} 			% para melhorias de justificação
\usepackage{hyperref}
\usepackage{subfig}
\usepackage{epigraph}
\usepackage{url}
\usepackage{placeins}
\usepackage{multirow}
\usepackage[figuresright]{rotating}
\usepackage{chemfig}
\usepackage{amsmath}
\usepackage{amssymb}
\usepackage{enumitem}
\usepackage{bigints}
\usepackage{listings}
\usepackage{etoolbox}
\usepackage[final]{pdfpages}
\usepackage{bigstrut}
\usepackage{makeidx}
\usepackage{float}

% ---



% ---
% Pacotes adicionais, usados apenas no âmbito do Modelo Canônico do abnteX2
% ---
\usepackage{lipsum}				% para geração de dummy text
% ---

% ---
% Pacotes de citações
% ---
\usepackage[brazilian,hyperpageref]{backref}	 % Paginas com as citações na bibl
% \usepackage[alf,abnt-emphasize=bf]{abntex2cite}  % Citações padrão ABNT
\usepackage[alf,abnt-etal-cite=2]{abntex2cite}  % Citações padrão ABNT

% ---------------------------------------------------------------------------------
%                          CONFIGURAÇÕES DOS PACOTES
% ---------------------------------------------------------------------------------

% ---
% Configurações do pacote backref
%
% Para desativar, tire o comentário de \begin{comment} e \end{comment} 
% das próximas linhas e comente a linha \usepackage[brazilian,hyperpageref]{backref}
% acima.
% ---

% \newcommand{\imprimirdata}{\data}


%\begin{comment}
% ---
% Configurações do pacote backref
% Usado sem a opção hyperpageref de backref
\renewcommand{\backrefpagesname}{Citado na(s) página(s):~}
% Texto padrão antes do número das páginas
\renewcommand{\backref}{}
% Define os textos da citação
\renewcommand*{\backrefalt}[4]{
	\ifcase #1 %
	Nenhuma citação no texto.%
	\or
	Citado na página #2.%
	\else
	Citado #1 vezes nas páginas #2.%
	\fi}%
% ---
%\end{comment}


% listagens
\definecolor{corComentario}{RGB}{150,150,150}
\definecolor{corString}{RGB}{206,123,0}
\definecolor{corPalavraChave}{RGB}{0,0,230}

\lstset{
	numbers=left,
	stepnumber=1,
	firstnumber=1,
	numberstyle=\footnotesize,
	extendedchars=true,
	breaklines=true,
	lineskip=0pt,
	frame=tb,
	basicstyle=\ttfamily\footnotesize,
	showstringspaces=false,
	stringstyle=\color{corString},
	commentstyle=\color{corComentario},
	keywordstyle=\color{corPalavraChave}
}

\newcommand{\graficoname}{Gráfico}
\newcommand{\graficorefname}{Gráfico}
\newcommand{\listofgraficosname}{Lista de gráficos}





% \newcolumntype{Y}{>{\centering\arraybackslash}X}

\newcommand{\ano}[1]{\def \oano {#1}}
\newcommand{\imprimirano}{\oano}

\newcommand{\mes}[1]{\def \omes {#1}}
\newcommand{\imprimirmes}{\omes}

\newcommand{\dia}[1]{\def \odia {#1}}
\newcommand{\imprimirdia}{\odia}

% \newcommand{\subtitulo}[1]{\def \osubtitulo {#1}}
% \newcommand{\imprimirsubtitulo}{\osubtitulo}

\newcommand{\area}[1]{\def \aarea {#1}}
\newcommand{\imprimirarea}{\aarea}

\newcommand{\disciplina}[1]{\def \adisciplina {#1}}
\newcommand{\imprimirdisciplina}{\adisciplina}

% \renewcommand{\coorientador}[1]{\def \ocoorientador {#1}}
% \renewcommand{\imprimircoorientador}{\ocoorientador}

\newcommand{\grau}[1]{\def \ograu {#1}}
\newcommand{\imprimirgrau}{\ograu}

\newcommand{\curso}[1]{\def \ocurso {#1}}
\newcommand{\imprimircurso}{\ocurso}

\newcommand{\campus}[1]{\def \ocampus {#1}}
\newcommand{\imprimircampus}{\ocampus}


% Área de Concentração: \imprimirarea}
% ---


% ---
% Configurações de aparência do PDF final
% ---

% alterando o aspecto da cor azul
\definecolor{blue}{RGB}{41,5,195}

% informações do PDF
\makeatletter
\hypersetup{
	%pagebackref=true,
	pdftitle={\@title}, 
	pdfauthor={\@author},
	pdfsubject={\imprimirpreambulo},
	pdfcreator={\@author},
	pdfkeywords={Palavra chave 1}{Palavra chave 2}{Palavra chave 3}{Palavra chave n}, 
	colorlinks=true,       		% false: boxed links; true: colored links
	linkcolor=blue,          	% color of internal links
	citecolor=blue,       		% color of links to bibliography
	filecolor=black,      		% color of file links
	urlcolor=blue,
	bookmarksdepth=4
}
\makeatother
% --- 


% ---
% Comandos do autor
% ---

% comando para inserir autor e ano
\newcommand{\citeauthorandyear}[1]{\citeauthoronline{#1} (\citeyear{#1})}

% --- 
% Espaçamentos entre linhas e parágrafos 
% --- 

% O tamanho do parágrafo é dado por:
\setlength{\parindent}{1.3cm}

% Controle do espaçamento entre um parágrafo e outro:
\setlength{\parskip}{0.2cm}  % tente também \onelineskip

\glsaddall

% ---
% compila os glossários e listas
% \makeglossaries

% ---
% compila o indice
% ---
\makeindex
% ---
%---


%%%%%%%%%%%%%%%%%%%%%%%%%%%%%%%%%%%%%%%%%%%%

% para o pacote "Lst Listing"

\newcommand{\source}[1]{\caption*{Fonte: {#1}}}

% Altera o nome padrão do rótulo usado no comando \autoref{}
\renewcommand{\lstlistingname}{Código}

% Altera o rótulo a ser usando no elemento pré-textual "Lista de código"
\renewcommand{\lstlistlistingname}{Lista de códigos}

%% Configuração para o ambiente de código (algorítmos)
\lstset{
%   numbers=left,
%   inputencoding=latin1,
%   basicstyle=\footnotesize\ttfamily,
%   keywordstyle=\color{blue},         
%   breaklines=true, 
%   showtabs=false,
%   showstringspaces=false,
%   numberstyle=\tiny\color{mygray}
   basicstyle=\fontsize{9}{11}\selectfont\ttfamily
}
%% 
%% exemplo da página https://latex.org/forum/viewtopic.php?t=24038
%%
