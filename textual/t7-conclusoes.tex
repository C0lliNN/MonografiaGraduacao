\chapter{Conclusão}
\label{cap:conclusão}

Este trabalho apresentou um estudo de caso sobre a utilização de \acrfull{ddd} e \acrfull{ams} no desenvolvimento de um sistema para uma locadora de veículos. O objetivo foi avaliar como essas tecnologias podem ser utilizadas para aumentar a escalabilidade e resiliência de sistemas complexos. Com base dos resultados obtidos com a execução dos testes de carga, é possível concluir que essa abordagem é eficaz para a construção de sistemas distribuídos. Além disso, nota-se como diferentes partes do sistema são expostos a tráfegos distintos, o que permite a escalabilidade independente de cada serviço.

Com a modelagem do sistema através de \acrshort{ddd}, foi possível definir limites claros entre os diferentes contextos do negócio com a utilização de \acrfull{bc}. Além disso, a utilização de \acrshort{ddd} permitiu a definição de um modelo de domínio rico e expressivo, que reflete de forma fiel as regras de negócio da locadora de veículos.

Por otro lado, o desenvolvimento deste estudo de caso trouxe diversos desafios como a necessidade de um maior esforço de \english{design upfront} para garantir a correta separação de serviços e a definição do domínio do negócio. Além disso, percebe-se uma maior complexidade para realização de testes e depuração de problemas, uma vez que o sistema é composto por diversos serviços independentes.

Este trabalho cumpriu com os objetivos propostos, na medida que apresentou uma estratégia para transformar requisitos funcionais em um \english{design} com \acrshort{ams} e \acrshort{ddd}, prover informações relevantes para definição dos estilo de comunicação adequado entre microsserviços e demonstrar desempenho e escalabilidade do sistema desenvolvido através de testes de carga.
