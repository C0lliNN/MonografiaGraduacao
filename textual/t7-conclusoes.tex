\chapter{Conclusão}
\label{cap:conclusão}
%%%%%%%%%%%ALTERADO%%%%%%%
Com o passar dos anos, o acúmulo de detritos de lixo causaram transtornos ao meio ambiente. A reciclagem de materiais surge como solução deste problema, e se torna essencial para subsistência da nossa sociedade.

A proposta deste trabalho, foi o desenvolvimento de uma ferramenta que possibilite a integração entre pessoas que realizam a separação dos materiais recicláveis, e cooperativas ou pessoas que trabalham com a coleta de forma independente utilizando o processamento e a facilidade do celular para a otimização dos serviços de coleta seletiva. A experimentação realista simulando o uso do sistema, foi bem aceita pelos usuários. Foi possível relacionar as visões dos separadores com os catadores.

Além disso, com a informação do tipo de material depositado pelo separador, o catador otimiza seu tempo, podendo ter um maior número de materiais recolhidos, e com a facilidade de escolher onde descartar o material separado, mais pessoas podem realizar a separação de materiais em suas rotinas. 

Com o protótipo de aplicação aplicativo torna possível minimizar os impactos ambientais, que o excesso de acúmulo de detritos de lixo trazem para a sociedade e ao meio ambiente.

Entende-se que como um primeiro passo para um produto maior, as várias funcionalidades idealizadas para este projeto foram implementadas, experimentadas de forma realística e bem aceitas, confirmando o sucesso deste projeto.

\section{Trabalhos Futuros}

Durante o desenvolvimento deste trabalho foram observadas algumas melhorias a serem implementadas, sendo assim vistas como trabalhos futuros. Dentre elas podemos citar:

\begin{itemize}
	\item Acrescentar todos os pontos de descarte no mapa;
	\item Otimizar a rota através de algoritmos;
	\item Aplicar filtros ao mapa, para que o catador possa visualizar somente os pontos pelo tipo de material selecionado;
	\item Enviar notificações quando forem inseridos novos pontos de descarte e quando for sinalizado o recolhimento de um material;
\setlength{\parskip}{9.96pt}
	\item Adicionar outros métodos de \textit{login}, como por exemplo: o \textit{Facebook, }o \textit{Gmail }e o número de telefone;
	\item Inserir uma rota dinâmica no programa, em função da localização instantânea do usuário.
\end{itemize}

 \glsaddall