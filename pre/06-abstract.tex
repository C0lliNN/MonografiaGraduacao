\begin{resumo}[Abstract]
 \begin{otherlanguage*}{english}

% O comando lipsum abaixo é um gerador automático de texto, e deve ser substituído pelo seu abstract.

Considering the popularization of cloud computing, the increasing demand for more frequent changes and the growth of information technology teams in large organizations, the traditional architecture for developing enterprise applications, based on monoliths, is confronted with significant challenges. In this scenario, the Microservices Architecture emerges as an alternative. At the same time, the Domain-Driven Design (DDD) methodology has been increasingly used to model complex business domains. These two strategies can be combined in the design and development of systems. On one side, microservices enable independent scalability, decoupled deployment, and the use of multiple technologies for certain use cases. On the other hand, DDD provides an approach to modeling the business domain, several patterns to solve recurring modeling problems, and ease of understanding and maintaining code. In addition, DDD is an excellent strategy for defining boundaries between microservices. However, the combined use of these two concepts faces nuances and different possibilities. Therefore, this work presents a realistic case study of vehicle retailer  with the use of these strategies aiming to offer a significant contribution to the understanding of the integrated application of these approaches.

\textbf{Keywords: } Microservices, Domain-Driven Design, Microservices Architecture.

\end{otherlanguage*}
\end{resumo}
