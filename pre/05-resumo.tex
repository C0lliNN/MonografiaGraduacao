\begin{resumo}

% O comando lipsum abaixo é um gerador automático de texto.
% Substitua-o pelo texto do seu resumo.
% Lembre-se: Um resumo deve ser um parágrafo único que apresente os seguintes tópicos:

% Contexto;
% Problema;
% Objetivo;
% Justificativa;
% Metodologia;
% Resultado;
% Conclusão.

Com a popularização da computação em nuvem, a crescente demanda de mudanças cada vez mais frequentes e o crescimento das equipes de tecnologia da informação em grandes organizações, a arquitetura tradicional para o desenvolvimento de aplicações corporativas, baseada em monólitos, é confrontada com desafios significativos. Nesse cenário, surge como alternativa a Arquitetura de Microsserviços. Paralelamente a isso, a abordagem \textit{Domain-Driven Design} (DDD) tem sido cada vez mais utilizada para modelagem de domínios de negócio complexos. Essas duas estratégias podem ser combinadas no \textit{design} e desenvolvimento de sistemas. De um lado, microsserviços possibilitam escalabilidade independente, implantação desacoplada e utilização de múltiplas tecnologias para determinados casos de uso. Por outro lado, DDD fornece uma abordagem para modelagem do domínio de negócio, diversos padrões para resolver problemas de modelagem recorrentes e facilidade de entendimento e manutenção de código. Além disso, DDD é uma excelente estratégia para definição de limites entre microsserviços. No entanto, a utilização combinada desses dois conceitos apresenta nuances e diferentes possibilidades. Assim, esse trabalho apresenta um estudo de caso realista de uma locadora de veículos com o emprego dessas estratégias visando oferecer uma contribuição significativa para a compreensão da aplicação integrada dessas abordagens.

\textbf{Palavras-chaves: } Microsserviços, \textit{Domain-Driven Design}, Arquitetura de Microsserviços.  

\end{resumo}


