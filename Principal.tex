% !TeX root = Principal.tex


\input{configuracao}
\newacronym{dvd}{DVD}{Digital Versatile Disc}
\newacronym{ddd}{DDD}{Domain-Driven Design}
\newacronym{soa}{SOA}{Service-Oriented Architecture}
\newacronym{poo}{POO}{Programação Orientada a Objetos}
\newacronym{cpu}{CPU}{Central Processing Unit}
\newacronym{ide}{IDE}{Integrated Development Environment}
\newacronym{ipc}{IPC}{Inter-Process Communication}
\newacronym{lgpd}{LGPD}{Lei Geral de Proteção de Dados Pessoais}
\newacronym{ohs}{OHS}{Open Host Service}
\newacronym{acl}{ACL}{Anti-Corruption Layer}
\newacronym{api}{API}{Application Programming Interface}
\newacronym{bc}{BC}{Bounded Context}
\newacronym{ams}{AMS}{Arquitetura de Microsserviços}
\newacronym{vo}{VO}{Value Object}
\newacronym{acid}{ACID}{Atomicidade, Consistência, Isolamento e Durabilidade}



% ---------------------------------------------------------------------
% Informações de dados para CAPA, FOLHA DE ROSTO e FOLHA DE ASSINATURAS
% ---------------------------------------------------------------------

% ---------------------------------------------------------------------
% Alterar os dados abaixo com os seus dados
%
% obs: Os nomes e instituições dos membros da banca deverão ser
%       alterados no arquivo 03-assinaturas.tex 
% ---------------------------------------------------------------------

\curso{Bacharelado em Sistemas de Informação}
\grau{Bacharel em Sistemas de Informação} 
\titulo{Desenvolvimento de Sistemas com Microsserviços e \english{Domain-Driven Design}: um estudo de caso realista}
\tipotrabalho{Trabalho de Conclusão}
\area{Tecnologia da Informação}

\autor{Raphael Collin Teles Manhães}
\orientador{Prof. D.Sc. Mark Douglas de Azevedo Jacyntho}

% caso não haja coorientador, comente a linha abaixo
% \coorientador{Profa. M.Sc. Nome da Professora Coorientadora}

\local{Campos dos Goytacazes-RJ}
\dia{29}
\mes{abril}
\ano{2024}
\data{Abril de \imprimirano}

\instituicao{Instituto Federal de Educação, Ciência e Tecnologia Fluminense}
\campus{Campos-Centro}

\preambulo{\imprimirtipotrabalho\ apresentado ao curso \imprimircurso~ do \imprimirinstituicao, como parte dos requisitos para a obtenção do título de \imprimirgrau.}

% ---------------------------------------------------------------------------------
%                                   INÍCIO DO DOCUMENTO
% ---------------------------------------------------------------------------------

\begin{document}

\pretextual

\input{pre/01-capa}

\input{pre/02-rosto}

\input{pre/03-assinaturas}


\begin{agradecimentos}

Em primeiro lugar, agradeço a Deus por ter me dado força e sabedoria para superar os desafios e dificuldades ao longo da minha jornada acadêmica.

Agradeço também a minha mãe que sempre me apoiou e incentivou a seguir em frente na vida acadêmica e profissional desde os primeiros momentos. Sou muito grato por todo o amor, carinho e dedicação que sempre me proporcionou. Se hoje estou aqui, é graças a você.

Agradeço ao meu orientador, Prof. Dr. Mark Douglas, por todo o apoio, orientação e ensinamentos durante suas aulas, nossas conversas e ao longo deste trabalho. Sua paciência, dedicação e conhecimento foram fundamentais para o desenvolvimento deste estudo.

Por fim, sou extramente grato a todos professores, autores, colegas e amigos que em algum momento me ensinaram algo e contribuíram para a minha formação acadêmica e profissional. Acredito firmemente que a educação é a chave para não apenas para atingir nossos objetivos pessoais, mas também para transformar nosso país e o mundo em um lugar mais próspero, solidário e justo. 

\end{agradecimentos}




\begin{resumo}

% O comando lipsum abaixo é um gerador automático de texto.
% Substitua-o pelo texto do seu resumo.
% Lembre-se: Um resumo deve ser um parágrafo único que apresente os seguintes tópicos:

% Contexto;
% Problema;
% Objetivo;
% Justificativa;
% Metodologia;
% Resultado;
% Conclusão.

Com a popularização da computação em nuvem, a crescente demanda de mudanças cada vez mais frequentes e o crescimento das equipes de tecnologia da informação em grandes organizações, a arquitetura tradicional para desenvolvimento de aplicações corporativas, baseada em monólitos, é confrontada com desafios significativos. Nesse cenário, surge como alternativa a Arquitetura de Microsserviços. Paralelamente a isso, a abordagem \textit{Domain-Driven Design} (DDD) tem sido cada vez mais utilizada para modelagem de domínios de negócio complexos. Essas duas estratégias podem ser combinadas no \textit{design} e desenvolvimento de sistemas. De um lado, microsserviços possibilitam escalabilidade independente, implantação desacoplada e utilização de múltiplas tecnologias para determinados casos de uso. Por outro lado, DDD fornece uma abordagem para modelagem do domínio de negócio, diversos padrões para resolver problemas de modelagem recorrentes e facilidade de entendimento e manutenção de código. Além disso, DDD é uma excelente estratégia para definição de limites entre microsserviços. No entanto, a utilização combinada desses dois conceitos apresenta nuances e diferentes possibilidades. Assim, esse trabalho apresenta um estudo de caso realista de uma locadora de veículos com o emprego dessas estratégias visando oferecer uma contribuição significativa para a compreensão da aplicação integrada dessas abordagens.

\textbf{Palavras-chaves: } Microsserviços, \textit{Domain-Driven Design} e Arquitetura de Software.  

\end{resumo}




\begin{resumo}[Abstract]
 \begin{otherlanguage*}{english}

% O comando lipsum abaixo é um gerador automático de texto, e deve ser substituído pelo seu abstract.

\lipsum*[10-12]

\textbf{Keywords: } Curabitur, Malesuada, MassaAndroid, Aliquam. 

 \end{otherlanguage*}
\end{resumo}


\input{pre/07-figuras}

% não é necessário alterar este arquivo

\listofquadros


% \input{pre/09-tabelas}

% \input{pre/10-codigos}

\input{pre/11-siglas}

\input{pre/11-simbolos}

\input{pre/12-sumario}

\textual

\chapter{Introdução}
\label{cap:introducao}

\section{Problema e contexto}

A abordagem convencional no desenvolvimento de sistemas corporativos, na qual toda a lógica de negócio, persistência e apresentação são encapsuladas em uma única aplicação (e um único executável), é a mais natural para construção desse tipo de sistema. Essa arquitetura, conhecida comumente como monólito, tem alcançado sucesso devido à sua simplicidade no desenvolvimento, testes, implantação e monitoramento, entre outros fatores. No entanto, com a popularização da computação em nuvem, a crescente demanda de mudanças cada vez mais frequentes e o crescimento das equipes de tecnologia da informação em grandes organizações, essa arquitetura é confrontada com desafios significativos. Inicialmente, destaca-se o acoplamento nos ciclos de mudanças, onde qualquer alteração em uma pequena parte requer a compilação e implantação de todo sistema. Além disso, há dificuldade em escalar horizontalmente apenas funcionalidades específicas que estão sendo mais demandas e existe complexidade na utilização de diferentes linguagens de programação em diversas áreas de um sistema \cite{microservices}. Por esses motivos, a adoção da \acrfull{ams} cresceu de forma exponencial nos últimos anos, com 77\% dos profissionais afirmando que suas organizações utilizam esse estilo arquitetural \cite{microserviceAdoption}.

Por outro lado, \english{\acrfull{ddd}} é uma abordagem para o desenvolvimento de software, na qual os elementos de software como pacotes, classes, interfaces, métodos e nomes de variáveis devem corresponder aos conceitos do domínio de negócio \cite{dddFowler}. Adicionalmente, o \acrshort{ddd} se configura como uma estratégia para harmonizar especialistas de negócios, desenvolvedores e projetistas, promovendo uma otimização nas interações ao reduzir o mapeamento entre termos de negócio e termos técnicos.

Em agosto de 2008, um grande incidente interrompeu as operações da \emph{Netflix}, empresa que nesse período alugava DVDs através de seu site na \english{web}. Uma corrupção no banco de dados impediu que a companhia pudesse realizar envios por três dias. A partir desse momento, os engenheiros perceberam que necessitavam reduzir pontos únicos de falha, tais como grandes bases de dados relacionais armazenadas em somente um \english{data center}. Em vez disso, optaram por adotar sistemas distribuídos na nuvem, escalonados horizontalmente, para garantir maior confiabilidade e minimizar o impacto das falhas \cite{netflixMigration}.

Essa transição resultou em vários benefícios para a \emph{Netflix}, destacando-se a conquista de uma disponibilidade de 'quatro noves' (99,99\%) e significativa redução de custos. Além disso, a \short{ams} permitiu que diferentes aplicações utilizassem diferentes tecnologias, facilitando contratações e extraindo o melhor de cada opção. Assim, em 2016 (8 anos após a migração) a empresa tinha 8 vezes mais assinantes, muito mais ativos, com o número de visualizações crescendo em três ordens de grandeza nesse período \cite{netflixMigration}.

A \acrshort{ams} e \acrshort{ddd} podem ser combinados no \english{design} e desenvolvimento de sistemas. De um lado, microsserviços possibilitam escalabilidade independente, implantação desacoplada e utilização de múltiplas tecnologias para determinados casos de uso. Por outro lado, \acrshort{ddd} fornece uma abordagem para modelagem do domínio de negócio, diversos padrões para resolver problemas de modelagem recorrentes e facilidade de entendimento e manutenção de código. Além disso, \acrshort{ddd} é uma excelente estratégia para definição de limites entre microsserviços. Ao modelar os serviços com base em \english{\acrfull{bc}}s coesos, é possível o desenvolvimento de novas funcionalidades mais rapidamente e também se torna mais simples a recombinação de microsserviços para fornecer novas funcionalidades aos usuários \cite{buildingMicroservices}.

\section{Justificativa}

Recentemente, observou-se um notável crescimento em popularidade e adoção da \acrshort{ams} e da abordagem \acrshort{ddd} no desenvolvimento de sistemas corporativos. Porém, essas tecnologias são utilizadas de maneira equivocada em muitas implementações devido à falta de compreensão das suas vantagens e desvantagens, entre outros fatores. Por exemplo, a equipe de desenvolvimento do \emph{Amazon Prime Video} recentemente anunciou uma migração de um serviço de monitoramento de vídeo, que foi inicialmente desenvolvido com microsserviços, para uma arquitetura monolítica com objetivo de aumentar a escalabilidade e reduzir custos \cite{amazonBackMigration}.

Paralelamente a isso, é possível identificar uma carência de estudos de caso realistas envolvendo a utilização simultânea da \acrshort{ams} e \acrshort{ddd}, especialmente no cenário nacional.  Enquanto algumas publicações concentram-se exclusivamente em um desses conceitos e outras apresentam exemplos simplificados, proporcionando uma compreensão limitada de como essas duas técnicas podem ser efetivamente empregadas em conjunto. 

Nesse contexto, este trabalho visa preencher essa lacuna ao oferecer uma contribuição significativa para a compreensão da aplicação integrada dessas estratégias no desenvolvimento de sistemas. Por meio de um estudo de caso realista, busca-se não apenas enriquecer o conhecimento acadêmico, mas também fornecer uma base sólida para a implementação prática dessas tecnologias em diversos setores da indústria.

\section{Objetivos}

\subsection{Objetivo Geral}
O objetivo geral deste trabalho de conclusão de curso é apresentar um estudo de caso realista integrando a \acrshort{ams} e \acrshort{ddd}, visando demonstrar como essas tecnologias podem ser utilizadas para aumentar a escalabilidade, desempenho e resiliência de sistemas complexos.


\subsection{Objetivos Específicos}
\begin{itemize}
\item Apresentar uma estratégia para transformar requisitos funcionais em um \english{design} com \acrshort{ams} e \acrshort{ddd}.
\item Prover informações relevantes para definição dos estilo de comunicação adequado entre microsserviços.
\item Demonstrar desempenho e escalabilidade do sistema desenvolvido através de testes de carga.
\end{itemize}

\section{Metodologia}
Pode-se observar na \autoref{fig:metodo_recurso} as etapas de execução dessa pesquisa. Inicialmente, o escopo é definido e o primeiro capítulo é elaborado. Em seguida, a fundamentação teórica com os conceitos-chave é construída. Posteriormente, se realiza um mapeamento da literatura buscando trabalhos similares. Por último, um estudo de caso realista com a utilização de microsserviços e diversos conceitos do \acrshort{ddd} é desenvolvido e um relatório é produzido. Por fim, é escrita a conclusão do trabalho.

% Fill vertical space
\vfill

\begin{figure}[ht!]
    \centering
    \caption{Etapas de desenvolvimento da pesquisa}
    \includegraphics[width=0.9\textwidth]{media/bpmn_metodo_recurso.png}
    \legend{Fonte: o autor}
    \label{fig:metodo_recurso}
\end{figure}

\section{Estrutura do Trabalho}

Este trabalho está dividido em sete capítulos.  O \autoref{cap:introducao} expõe o contexto do estudo, as justificativas desta pesquisa e os objetivos a serem atingidos. O \autoref{cap:fundamentacao} apresenta conceitos de \acrshort{ddd}, \acrshort{ams} e afins. O \autoref{cap:trabalhos} expõe o protocolo e o resultado do mapeamento da literatura sobre a utilização combinada das estratégias mencionadas. Da mesma forma, o \autoref{cap:estudo_caso1} descreve os requisitos, métodos e organização do estudo de caso. Em seguida, o \autoref{cap:estudo_caso2} apresenta o estudo de caso desenvolvido, incluindo o \english{design} do sistema, trechos de código chave da implementação. Posteriormente, o \autoref{cap:resultados} apresenta os resultados obtidos com a execução dos testes de carga e uma discussão sobre os desafios e benefícios da utilização de \acrshort{ddd} e \acrshort{ams}. Por fim, o \autoref{cap:conclusão} apresenta as conclusões obtidas com o desenvolvimento deste trabalho.

\chapter{Fundamentação Teórica}
\label{cap:fundamentacao}

Este capítulo apresenta os conceitos de Sistemas Distribuídos, \acrfull{soa} Microsserviços, \acrfull{ddd} e Arquitetura Hexagonal.

\section{Sistema Distribuído} 
\label{section:sistemas_distribuidos}

Um sistema distribuído é um conjunto de componentes independentes entre si que se apresenta aos seus usuários de maneira transparente como um sistema único \cite{tanenbaum2010sistemas}. São sistemas compostos por diversas partes cooperantes, que são executadas geralmente em processos diferentes interconectados por rede a fim de realizar alguma tarefa. 

Diferentemente de sistemas monolíticos, nos quais todos os módulos estão interligados em uma única unidade, os sistemas distribuídos dividem o processamento entre as partes. Essa abordagem traz vantagens significativas, como a possibilidade de escalabilidade horizontal e maior flexibilidade para atualizar e substituir componentes individuais sem afetar o sistema como um todo e o compartilhamento de recursos.

Um dos principais desafios desse tipo de sistema é a concorrência e coordenação entre os componentes. Nessa abordagem, vários componentes podem estar operando simultaneamente e em diferentes locais físicos, o que gera problemas de concorrência, onde múltiplos processos tentam acessar os mesmos recursos compartilhados ou dados simultaneamente. Por esse motivo, torna-se essencial a coordenação entre esses processos para evitar conflitos e garantir a consistência dos dados.

\subsection{\english{\acrfull{soa}}} 
A arquitetura orientada a serviços ou \acrfull{soa} é estilo arquitetural no qual um sistema é dividido em componentes de software chamados de serviços, responsáveis por satisfazer uma necessidade de negócio específica.

As políticas, práticas e \english{frameworks} que permitem que a funcionalidade da aplicação seja fornecida e consumida como conjuntos de serviços publicados em uma granularidade relevante para o consumidor do serviço. Os serviços podem ser invocados, publicados e descobertos, e são abstraídos da implementação usando uma única forma de interface baseada em padrões \cite{understandingSOA}.

As principais vantagens da \acrshort{soa} são: reusabilidade dos serviços, simplicidade de manutenção, facilidade de escalar os serviços, maior disponibilidade do sistema e possibilidade de utilização de diferentes tecnologias nos serviços.

Porém, essa arquitetura possui uma maior complexidade para \english{design}, desenvolvimento, teste e implantação. Além desses fatores, é importante ressaltar um nível maior de dificuldade para depurar erros e encontrar profissionais qualificados para atuar nessa área.

\section{Microsserviços} 
A \acrfull{ams} é um subtipo da \acrshort{soa}. No entanto, trata-se de uma subcategoria que orienta como as fronteiras entre os serviços devem ser traçadas e que possui como ponto central a independência de implantação entre os serviços \cite{buildingMicroservices}.

Esse estilo arquitetural foca na criação de um sistema robusto, escalável e de fácil manutenção, desenvolvido através da composição de serviços pequenos, flexíveis e que resolvam uma necessidade negócio específica \cite{buildingMicroservices}. Além disso, cada microsserviço deve possuir sua própria base de dados. A troca de informações entre os componentes deve ser realizada através das interfaces fornecidas.

A implantação de um sistema com a utilização de microsserviços aumenta a resiliência geral do sistema, visto que possui melhor isolamento de falhas. Caso ocorra um mal-funcionamento em algum componente, somente uma parte das funcionalidades ficará indisponível. Tratando-se de um monólito, por outro lado, grande parte ou a totalidade dos recursos são afetados, pois todo o sistema é acoplado em uma única unidade.

Cada serviço na \acrshort{ams} pode ser escalado de maneira independente de outros serviços utilizando, por exemplo, escalonamento horizontal ou vertical \cite{richardson2018microservices}. Ademais, para cada serviço pode ser selecionado o \english{hardware} mais adequado para seu tipo de processamento. Dessa forma, uma parte do sistema que requer maior utilização da \acrshort{cpu} pode ser implantada em uma infraestrutura diferente de outra secção que necessita realizar mais operações de entrada e saída.

Outro benefício dessa arquitetura é o fato de cada serviço ser relativamente pequeno, sendo, assim, de fácil compreensão por todos que trabalham na base de código. Adicionalmente, um serviço pequeno possibilita um bom desempenho da \acrshort{ide} e também pode ser executado localmente de maneira rápida. Esses fatores contribuem para um aumento de produtividade da equipe \cite{richardson2018microservices}.

Por outro lado, com um sistema composto de múltiplos componentes, torna possível a utilização de diferentes tecnologias dentro da cada serviço. Dessa forma, a ferramenta mais apropriada para resolver a necessidade específica do serviço pode ser selecionada, eliminando a obrigatoriedade de escolher uma única tecnologia para todo o sistema \cite{buildingMicroservices}.

\subsection{Comunicação entre microsserviços}
Quando um sistema é composto por um único monólito, a comunicação entre os diferentes módulos é simplificada, pois se trata de um único processo ao nível de sistema operacional e, portanto, compartilha a área da memória principal. No entanto, microsserviços são implantados em diferentes processos(usualmente em diferentes computadores), assim se faz necessária a utilização de algum mecanismo de comunicação entre processos, também conhecido como \acrfull{ipc}.

Existe no mercado uma série de tecnologias, protocolos e ferramentas que viabilizam a comunicação entre microsserviços. Porém, antes de selecionar o mecanismo de \acrshort{ipc}, é importante compreender os estilos de interações entre sistemas \cite{richardson2018microservices}. Essa abordagem coloca foco nos requisitos da aplicação, no lugar de detalhes de um mecanismo específico.

Os modos de interação entre um cliente e um serviço podem ser classificados em duas dimensões: a quantidade de serviços que processam a requisição e a sincronicidade. No que diz respeito à primeira dimensão, existem duas opções: \textbf{"um para um"}, em que cada requisição é processada por um único serviço; e \textbf{"um para muitos"}, em que cada requisição pode ser processada por vários serviços. Por outro lado, a comunicação pode ser \textbf{síncrona}, onde o cliente espera uma resposta imediata e geralmente aguarda essa resposta para continuar seu processamento; ou \textbf{assíncrona}, onde o cliente envia a requisição e continua seu processamento sem esperar por uma resposta \cite{richardson2018microservices}.

\begin{quadro}[H]
\centering
\caption{Estilos de comunicação entre microsserviços}
\setlength{\tabcolsep}{0.8em} % for the horizontal padding
\renewcommand{\arraystretch}{1.5}% for the vertical padding
\begin{tabular}{p{1.5in}p{1.5in}p{1.5in}}
\hline

%row no:1
\multicolumn{1}{|p{1.5in}}{\textbf{}} & 
\multicolumn{1}{|p{1.5in}}{\textbf{Um para um}} & 
\multicolumn{1}{|p{1.5in}|}{\textbf{Um para muitos}} \\
\hhline{---}

%row no:2
\multicolumn{1}{|p{1.5in}}{\textbf{Síncrona}} & 
\multicolumn{1}{|p{1.5in}}{\english{Request/Response}} & 
\multicolumn{1}{|p{1.5in}|}{-} \\
\hhline{---}

%row no:3
\multicolumn{1}{|p{1.5in}}{\textbf{Assíncrona}} & 
\multicolumn{1}{|p{1.5in}}{\english{Asynchronous request/response} e \english{One-way notifications}} & 
\multicolumn{1}{|p{1.5in}|}{\english{Publish/subscribe} \english{Publish/async responses}} \\
\hhline{---}
\end{tabular}
\label{quad:estilos_comunicacao}
\fonte{\citeonline{richardson2018microservices}.}
\end{quadro}

No \autoref{quad:estilos_comunicacao}, observam-se alguns estilos de comunicação categorizados pelas dimensões anteriormente mencionadas. No \autoref{quad:padroes_um_para_um} são apresentados os diferentes tipos de comunicação \textbf{"um para um"}:

\begin{quadro}[H]
\centering
\caption{Padrões de comunicação do tipo \textbf{"um para um"}}
\setlength{\tabcolsep}{0.8em} % for the horizontal padding
\renewcommand{\arraystretch}{1.5}% for the vertical padding
\begin{tabular}{|p{1.2in}|p{3.5in}|}
\hline

\textbf{Nome} & \textbf{Descrição} \\ \hline
\english{Request/Response} & O serviço cliente realiza uma requisição a outro serviço e espera(geralmente bloqueando sua execução) por uma resposta. \\ \hline
\english{Asynchronous request/response} & O serviço cliente envia uma solicitação a um serviço, que responde de forma assíncrona. O cliente não fica bloqueado enquanto espera, pois o serviço pode não enviar a resposta por um longo período.\\ \hline
\english{One-way notifications} & O serviço cliente envia uma requisição, mas não espera nenhum tipo de resposta do serviço que processará a requisição. \\ \hline

\end{tabular}
\label{quad:padroes_um_para_um}
\fonte{\citeonline{richardson2018microservices}.}
\end{quadro}

Frequentemente, em ambientes de sistemas distribuídos, há a necessidade de vários serviços consumirem uma mensagem produzida por um determinado serviço. Nesse sentido, no \autoref{quad:padroes_um_para_muitos} são listadas algumas opções de padrões para realizar comunicações do tipo \textbf{"um para muitos"}:

\begin{quadro}[H]
\centering
\caption{Padrões de comunicação do tipo \textbf{"um para muitos"}}
\setlength{\tabcolsep}{0.8em} % for the horizontal padding
\renewcommand{\arraystretch}{1.5}% for the vertical padding
\begin{tabular}{|p{1.2in}|p{3.5in}|}
\hline

\textbf{Nome} & \textbf{Descrição} \\ \hline
\english{Publish/Subscribe} & Um cliente publica uma mensagem, que é consumida por zero ou mais serviços.. \\ \hline
\english{Publish/async responses} & Um cliente publica uma mensagem contendo uma requisição e espera por um determinado período por respostas. \\ \hline

\end{tabular}
\label{quad:padroes_um_para_muitos}
\fonte{\citeonline{richardson2018microservices}.}
\end{quadro}

\subsection{Estratégias para delimitação de microsserviços}
Uma decisão de grande impacto no projeto de sistema que utilize a \acrshort{ams} é o escopo de cada serviço. Em outras palavras, que funcionalidades serão parte de um determinado serviço e quais serão delegadas a outro componente.

\citeonline{buildingMicroservices} argumenta que a possibilidade de alterar um microsserviço de maneira isolada é fundamental para essa arquitetura. Além disso, \acrshort{ams} é apenas outra forma de decomposição modular, afirma o autor. Assim, os conceitos de \english{information hiding}, coesão e acoplamento podem ser aplicados nesse contexto para criar as delimitações eficientes entre os serviços.

\subsubsection{\english{Information Hiding}}
\english{Information hiding} é uma abordagem para desenvolvimentos de módulos, na qual se busca esconder o máximo de detalhes possíveis detrás da interface de um módulo \cite{buildingMicroservices}. Dessa forma, os benefícios esperados com a construção de módulos de alta qualidade são:
\begin{itemize}
    \item \textbf{Tempo de desenvolvimento aprimorado:} com a possibilidade de módulos serem construídos de forma independente, o desenvolvimento pode ser realizado de maneira paralela.
    \item \textbf{Compreensibilidade:} cada módulo pode ser entendido de maneira isolado. Assim, se torna mais fácil compreender o que o sistema como um todo faz.
    \item \textbf{Flexibilidade:} módulos podem ser modificados de maneira independente, possibilitando, dessa forma, alterações em uma parte do sistema sem a necessidade de alterações em outros módulos.
\end{itemize}
\cite{Parnas2012}.

\subsubsection{Coesão}
Coesão diz respeito à medida que os elementos dentro de um módulo estão organicamente relacionados entre si \cite{Yourdon1979}. Outra definição possível é código que, alterado conjuntamente, deve estar localizado no mesmo lugar \cite{buildingMicroservices}. 

No contexto da \acrshort{ams}, a adição ou atualização de funcionalidades deve, idealmente, envolver um único serviço. Dessa forma, é possível disponibilizar novos recursos aos usuários de maneira mais ágil \cite{buildingMicroservices}. Por esse motivo, buscar alta coesão é importante no momento da definição do escopo de cada serviço. O objetivo principal é evitar que sejam necessárias alterações em múltiplos serviços no desenvolvimento de funcionalidades para o sistema.

\subsubsection{Acoplamento}
Quando serviços são fracamente acoplados, uma alteração em um serviço não requer alteração no outro \cite{buildingMicroservices}. A grande vantagem da \acrshort{ams} é a possibilidade de realizar mudanças em partes do sistema de maneira isolada. Nesse sentido, a projeção de serviços com alta coesão e fraco acoplamento é essencial para que os objetivos da utilização dessa tecnologia sejam atingidos.

Nem todo acoplamento é necessariamente maléfico. De fato, algum nível de acoplamento entre os serviços é inevitável. Além disso, há diferentes tipos de acoplamento listados no \autoref{quad:tipos_acoplamento}.

\begin{quadro}[H]
\centering
\caption{Tipos de acoplamento}
\setlength{\tabcolsep}{0.8em} % for the horizontal padding
\renewcommand{\arraystretch}{1.5}% for the vertical padding
\begin{tabular}{|p{1.2in}|p{3.5in}|}
\hline

\textbf{Tipo} & \textbf{Descrição} \\ \hline
\english{Domain Coupling} & Descreve uma situação em que um microsserviço necessita interagir com outro serviço para fazer uso de sua(s) funcionalidade(s). Na \acrshort{ams}, esse tipo de iteração é quase inevitável. No entanto, é necessário que cada serviço exponha o mínimo de detalhes possível (\english{information hiding}) para não gerar um nível de acoplamento muito alto. \\ \hline
\english{Pass-Through Coupling} & Acontece quando um microsserviço envia dados a outro serviço somente porque essas informações são necessárias por um terceiro serviço. Trata-se de um dos tipos de acoplamento mais problemáticos, pois implica que o cliente da primeira requisição conheça tanto o serviço que inicialmente recebe a requisição, quanto o que irá, de fato, utilizar os dados. \\ \hline
\english{Common Coupling} & Ocorre quando dois ou mais microsserviços compartilham o mesmo conjunto de dados. O exemplo mais comum desse tipo de acoplamento é a utilização de uma base de dados compartilhada. O principal problema dessa abordagem é o fato de modificações ou falhas na base de dados compartilhadas afetar mais de um serviço. \\ \hline
\english{Content Coupling} & Descreve uma situação em que o serviço acessa e modifica estados internos de outro serviço. É semelhante ao \english{Common coupling}, porém nesse caso as linhas de responsabilidades de cada serviço se tornam confusas. Assim, desenvolvedores encontram grandes dificuldades para realizar alterações. \\ \hline

\end{tabular}
\label{quad:tipos_acoplamento}
\fonte{\citeonline{buildingMicroservices}.}
\end{quadro}

\subsubsection{\english{\acrfull{bc}}}
Visando a construção de serviços com alta coesão e baixo acoplamento, portanto estáveis, os conceitos de \acrshort{ddd} podem ser utilizados. Uma abordagem interessante e comumente aplicada é mapear um \nameref{section:bounded_context} para cada serviço \cite{buildingMicroservices}. Como os \hyperref[section:bounded_context]{Bounded Contexts} representam uma parte do domínio onde um modelo se aplica, eles permitem que microsserviços sejam modelados ao redor de uma secção coesa e completamente funcional que ofereça funcionalidades de negócio. 

Outra opção para modelagem de microsserviços é mapear cada \nameref{section:aggregate} para um serviço. Essa estratégia oferece um nível de granularidade menor que um \nameref{section:bounded_context}, visto que este geralmente engloba múltiplos \nameref{section:aggregate}s. No entanto, \citeonline{buildingMicroservices} sugere recorrer primeiro a delimitações maiores na construção de microsserviços, englobando um (ou até mais) \nameref{section:bounded_context}. Dessa forma, caso a equipe sinta necessidade de subdividir o serviço, poderá fazê-lo sem complicações.

\subsubsection{Alternativas}
Apesar de \acrfull{ddd} ser uma excelente maneira de delimitar microsserviços, não é a única técnica disponível \cite{buildingMicroservices}. São listadas a seguir algumas alternativas frequentemente utilizadas:
\begin{itemize}
    \item \textbf{Volatilidade:} decomposição baseada em volatilidade refere-se a uma abordagem, na qual partes do sistema que são frequentemente alteradas são extraídas em seus próprios serviços, de forma que possam ser trabalhadas de maneira mais efetiva. Essa estratégia por ser útil em organizações com ritmo de mudanças acelerado. No entanto, se o objetivo da utilização da \acrshort{ams} é a possibilidade de escalar partes do sistema de maneira independente, decomposição baseada em volatilidade não atingirá a meta.
    \item \textbf{Dados:} a natureza das informações que necessitam ser armazenadas pode direcionar a estratégia de decomposição. Por exemplo, é muito usual restringir quais microsserviços podem acessar informações de identificação pessoal para cumprir com legislações como a \acrfull{lgpd}.
    \item \textbf{Tecnologia:} A necessidade de se utilizar diferentes tecnologias também pode ser um fator a se considerar na definição do escopo de um microsserviço. Se há a necessidade, por exemplo, de alto desempenho em uma parte do sistema que demande a utilização de uma linguagem de programação diferente, este pode ser um ponto importante para delimitação do serviço.
\end{itemize}
\cite{buildingMicroservices}.

É importante mencionar que essas opções não são mutualmente exclusivas. Em um sistema real, provavelmente uma mistura de estratégias necessitará ser aplicada para satisfazer os requisitos funcionais e não-funcionais.

\subsection{Desafios na implementação de microsserviços}
Certamente, não há nenhuma tecnologia que não possua limitações. Isso não é diferente com a \acrlong{ams} \cite{richardson2018microservices}. 

Primeiramente, como microsserviços baseiam-se no conceito de \nameref{section:sistemas_distribuidos}, uma série de novos problemas estão presentes nessa arquitetura. Como os serviços necessitam utilizar algum mecanismo de \acrshort{ipc}, há um grande aumento na complexidade do ciclo de desenvolvimento como um todo. \acrshort{ide}s e outras ferramentas de desenvolvimento possuem maior foco no desenvolvimento de aplicações monolíticas. Adicionalmente, escrever testes automatizados envolvendo múltiplos serviços é uma tarefa complexa. Além disso, a \acrshort{ams} introduz dificuldades significantes de operação, como necessidade de alto nível de automação e ferramentas de monitoramento avançadas. Assim, uma equipe técnica altamente qualificada se torna necessária, podendo inclusive ser útil a contratação de consultorias especializadas nessa tecnologia \cite{richardson2018microservices}.

Por outro lado, a \acrfull{ams} engloba desafios para manter a consistência dos dados. Enquanto em um sistema monolítico, dados são usualmente armazenados em uma única base de dados relacional, que oferece suporte a transações \acrshort{acid}, em um sistema distribuído, no qual múltiplos banco de dados - possivelmente de tipos e fornecedores diferentes - são empregados, esse tipo de segurança não pode ser atingido facilmente \cite{buildingMicroservices}. Transações distribuídas ou \english{Sagas} podem ser aplicadas para reduzir problemas de consistência, no entanto, essas técnicas adicionam grande complexidade ao sistema \cite{buildingMicroservices}.

\section{\english{Domain-Driven Design (DDD)}} 
\english{\acrfull{ddd}} é uma abordagem para o desenvolvimento de software focada na construção de um modelo de domínio que representa de maneira sofisticada os processos e regras de negócio do sistema em questão \cite{dddFowler}. \acrshort{ddd} é muito efetivo para desenvolver sistemas com regras de negócio complexas, que possuem, por exemplo, diversos atores, entidades e validações. A \acrfull{poo} é geralmente empregada para a implementação do modelo de domínio, visto que possui uma série de recursos que facilitam a modelagem de objetos do mundo real \cite{evans2004ddd}.

Um ponto central da metodologia \acrshort{ddd} no desenvolvimento de softwares complexos, é a utilização de uma linguagem ubíqua que insere terminologia e jargões do domínio em componentes de software como atributos, métodos, classes, interfaces e pacotes \cite{dddFowler}. Assim, tanto especialistas de negócio quanto desenvolvedores e projetistas compartilham um vocabulário comum que facilita na comunicação e reduz a necessidade de realizar o mapeamento entre termos técnicos e termos do domínio. Além disso, a estrutura de sistema passa a representar a estrutura do negócio, tornando-se, assim, mais fácil de ser compreendida e modificada.

O nome dessa estrategia vem de um livro publicado em \citeyear{evans2004ddd} por Eric Evans que descreve a abordagem, uma série de conceitos e um catálogo de  padrões comumente utilizados para resolução de problemas de modelagem de domínios. Os mais relevantes dentre estes são descritos a seguir.

\subsection{Modelo e Domínio}
Os conceitos de modelo e domínio, no contexto da engenharia de software, são fundamentais para o entendimento do \acrshort{ddd}. Modelo é uma simplificação, uma interpretação da realidade que abstrai os aspectos relevantes para resolver o problema em questão e ignora detalhes superficiais \cite{evans2004ddd}. Da mesma forma, todo sistema computacional está relacionado a alguma atividade ou interesse do usuário final. Assim, a área para qual o usuário utiliza o sistema é domínio do software \cite{evans2004ddd}. Alguns exemplos de domínios comuns são: financeiro, médico, logística e \english{marketplace}.

\subsection{Entidade}
No contexto da \acrfull{poo}, muitos objetos não são fundamentalmente definidos por seus atributos, mas por um fio de continuidade e identidade \cite{evans2004ddd}. Eles podem ter distintas representações como, por exemplo, em memória ou em um banco de dados. No entanto, eles necessitam ser distinguíveis mesmo se todas suas propriedades forem iguais. Em um domínio financeiro, por exemplo, uma transação composta por valor, data e usuário necessita ser diferenciável de outra transação com as mesmas características. Por outro lado, para uma classe fictícia chamada \english{Money} composta por valor e moeda, dois objetos com os mesmos atributos podem ser geralmente considerados iguais, sem gerar corrupções de dados ou impactos para os usuários finais do sistema. Em resumo, quando um objeto definido principalmente por sua identidade, ele é chamado de Entidade \cite{evans2004ddd}.

\subsection{\english{\acrfull{vo}}}
Muitos objetos não possuem identidade conceitual, sendo somente utilizados para descrever características de outro objeto. São objetos que descrevem coisas, classificados como \acrfull{vo} \cite{evans2004ddd}. Para um sistema de uma loja de roupas, por exemplo, um objeto que representa cor de cada peça formado pelas propriedades \english{red}, \english{green} e \english{blue}, não possui identidade. Não necessita ser diferenciável de outro objeto com as mesmas propriedades. Este objeto somente descreve uma característica de uma peça de roupa. É importante ressaltar, porém, que a diferenciação entre Entidade e \english{\acrlong{vo}} depende do contexto do software. Para um sistema de software de computação gráfica, a cor é um conceito chave no sistema e pode possuir identidade.

Pode-se pensar que é mais vantajoso sempre projetar os objetos do sistema como Entidade, pois adicionaria identidade, facilitaria o rastreamento de mudanças e agregaria mais recursos ao objeto. No entanto, a adição de identidade para objetos que não necessitam desta classificação pode prejudicar o desempenho do sistema, adicionar mais esforço de análise e aumentar a complexidade do modelo de domínio \cite{evans2004ddd}. Por esse motivo, recomenda-se a utilização de \acrshort{vo} sempre que for possível, pois se tratam de objetos mais fáceis de gerenciar.

\subsection{Serviço}
Em muitos casos, há operações importantes no modelo que não pertencem conceitualmente a nenhuma Entidade ou \acrshort{vo}. Dessa forma, surge o conceito de Serviço: uma operação fornecida como uma interface autônoma no modelo, sem encapsular estado como outros tipos de objeto \cite{evans2004ddd}.

Um erro comum é a abusar da utilização de Serviços, desistindo de encontrar o objeto apropriado para uma operação de domínio. Da mesma forma, a inserção de um método que não está alinhado com a definição do objeto resulta na perda de coesão desse objeto, tornando-o mais difícil de compreender e refatorar \cite{evans2004ddd}. Considerando esses pontos, um serviço deve ser utilizado quando um processo ou transformação relevante no modelo de domínio não faz parte da responsabilidade natural de uma Entidade ou \acrshort{vo}.

Além disso, \citeonline{evans2004ddd} destaca três características de um bom Serviço:
\begin{itemize}
    \item A operação não possui estado (\english{stateless}).
    \item A interface é defina em termos de outros elementos do modelo de domínio.
    \item A operação se relaciona a um conceito do modelo que não é parte natural de uma Entidade ou \acrshort{vo}.
\end{itemize}

\subsection{\textit{Aggregate}}
\label{section:aggregate}
Um \english{Aggregate} é um conjunto de objetos associados, tratados como uma única unidade para o proposito de alterações de dados \cite{evans2004ddd}. Em um sistema de grande porte, é difícil garantir a consistência de alterações em objetos com relacionamentos complexos. As invariantes que precisam ser mantidas geralmente envolvem um grupo de objetos intimamente associados, ao invés de objetos isolados. Assim, o padrão \english{Aggregate} propõe delimitações definidas e restrições de acesso visando o melhor tratamento das associações entre objetos.

Cada \english{Aggregate} possui uma raiz e uma delimitação, que define o que está dentro do \english{Aggregate} \cite{evans2004ddd}. A raiz é a única entidade global - que possui identidade em todo o sistema - contida dentro do \english{Aggregate}, também conhecida como \english{Aggregate root} \cite{evans2004ddd}. Objetos fora da delimitação do \english{Aggregate} só podem possuir referências a raiz e não a outros objetos dentro do \english{Aggregate}. Excluindo a raiz, outras entidades somente possuem identidade local, necessitando somente ser distinguíveis dentro dos limites do \english{Aggregate}, visto que nenhum objeto de fora consegue obtê-las desassociadas da entidade raiz \cite{evans2004ddd}.

\subsection{\acrfull{bc}}
\label{section:bounded_context}
Em grandes projetos, muitos modelos coexistem em diferentes contextos. Quando a base de código, no entanto, combina diferentes modelos sem que haja uma separação clara, o \english{software} se torna defeituoso, frágil e difícil de ser compreendido \cite{evans2004ddd}. Assim, é necessário especificar em qual o contexto um determinado modelo está sendo aplicado.  Assim, um \english{\acrfull{bc}} delimita a aplicabilidade de um modelo, de tal forma que os membros da equipe tenham um entendimento claro do que precisa ser consistente e a relação do modelo com outros contextos. Um \acrshort{bc} engloba tanto componentes de modelo como \english{Aggregates}, Entidades, \english{\acrlong{vo}}s e Serviços, quanto detalhes técnicos como persistência. É uma parte completa e funcional de software sob responsabilidade de uma única equipe.

Por exemplo, um \english{software} para um \english{e-commerce} possui um módulo que lida com pedidos, em que cada pedido contém um ou mais produtos. Da mesma forma, outra equipe é responsável por um módulo de realiza o gerenciamento de produtos em estoque. O conceito de um produto para um sistema de pedidos é distinto de uma aplicação de estoque. Enquanto este pode ter um enfoque no armazém, na categoria e no estoque do produto, aquele está mais preocupado com o preço, nome e disponibilidade. Assim, ao tratar esses dois conceitos da maneira igual, confusões e defeitos são gerados.

\section{Arquitetura Hexagonal} 
A arquitetura hexagonal, também conhecida como \english{Ports and Adapters}, possui como objetivo isolar regras de negócio de agentes externos \cite{cockburn2005}. Nesse sentido, o que está dentro do hexágono representa o modelo de domínio. Do mesmo modo, componentes de fora representam detalhes de implementação, como persistência, interface de usuário, envio de \english{e-mails}, entre outros. Um ponto crucial nessa arquitetura é a direção das dependências: o que está dentro do hexágono não pode depender do que está fora. Assim, é a garantida a independência e o isolamento de regras de negócio das dependências técnicas.

\begin{figure}[!ht]
    \centering
    \caption{Arquitetura Hexagonal}
    \includegraphics[width=0.75\textwidth]{media/hexagonal_architecture.png}
    \legend{Fonte: o autor}
    \label{fig:arquitetura_hexagonal}
\end{figure}

Na \autoref{fig:arquitetura_hexagonal}, pode-se observar que a injeção de agentes externos, como persistência em um banco de dados, é realizada por meio de portas definidas dentro do núcleo da aplicação. As portas, nessa arquitetura, são geralmente representadas por interfaces em linguagens orientadas a objetos. Então, um adaptador fornece uma implementação de uma tecnologia específica para a porta. Assim, implementações para tecnologias específicas são definidas em função do modelo de domínio.

Outros benefícios importantes, além do isolamento das regras de negócio, são: facilidade de testar o núcleo da aplicação com a utilização de \english{mocks} e \english{stubs}; possibilidade de utilização da mesma aplicação via interfaces de usuário, testes automatizados e \english{batches}.

Enquanto a \acrfull{ams} visa organizar um sistema como um conjunto de serviços independentes, cada um executando uma função específica, a arquitetura hexagonal concentra-se na organização interna de um serviço, propondo uma separação clara entre a lógica de negócios e as implementações técnicas, usando portas e adaptadores. Por outro lado, \acrshort{ddd} apresenta-se como uma excelente abordagem para desenvolvimento do núcleo da aplicação na arquitetura hexagonal. Assim, esses três conceitos: microsserviços, arquitetura hexagonal e \acrshort{ddd} podem ser combinados para o desenvolvimento de sistemas altamente coesos, desacoplados, escaláveis, inteligíveis e sustentáveis.






\chapter{Trabalhos Relacionados}
\label{cap:trabalhos}
Este capítulo apresenta os trabalhos relacionados ao objeto de pesquisa obtidos utilizando o protocolo descrito no \autoref{cap:metodologia}.

\section{Caracterização dos estudos}

\begin{figure}
\centering
\caption{Número de publicações ao longo dos anos}
\begin{tikzpicture}
\begin{axis}[
    width=0.8\textwidth,
    height=0.6\textwidth,
    xlabel={Ano},
    ylabel={Número de publicações},
    xtick=data, % Use data points for x-axis ticks
    xticklabel style={/pgf/number format/1000 sep=}, % Remove comma from tick labels
    ymajorgrids=true, % Show y-axis grid lines
    grid style=dashed, % Style of grid lines
    legend pos=north west, % Position of the legend
]
\addplot[
    smooth, % Show only the dots without connecting lines
    mark=*,
    mark options={scale=1.5}, % Adjust dot size
] table [x=Year, y=Publications, col sep=comma] {media/publications_per_year.csv};
\end{axis}
\end{tikzpicture}
\legend{Fonte: \english{Scopus}}
\label{fig:publications_per_year}
\end{figure}

Na \autoref{fig:publications_per_year}, pode-se observar a quantidade de publicações por ano, encontradas a partir da expressão de busca descrita na seção \ref{section:string_busca}. No total, foram obtidos 65 resultados. Percebe-se, claramente, um aumento exponencial no número de publicações sobre o tema, principalmente a partir de 2019, demonstrando crescimento no interesse de se realizar pesquisas sobre o assunto.

\begin{figure}
    \centering
    \caption{Publicações por localização}
    \pgfplotstableread[col sep=comma]{media/publications_per_territory.csv}\datatable

    \begin{tikzpicture}
        \begin{axis}[
            xbar,
            bar width=0.5cm,
            height=0.6\textwidth,
            width=0.8\textwidth,
            symbolic y coords={Alemanha, China, Áustria, Itália, Suíça, Taiwan, Estados Unidos, Indonésia, Brasil, Colômbia},
            ytick=data,
            xlabel={Publicações},
            ylabel={Localização},
            nodes near coords,
            nodes near coords align={horizontal},
            ]
            \addplot table[x=Publications, y=Territory] {\datatable};
        \end{axis}
    \end{tikzpicture}
    \legend{Fonte: \english{Scopus}}
    \label{fig:publications_per_territory}
\end{figure}

A Alemanha lidera em número de publicações, como pode ser visualizado na \autoref{fig:publications_per_territory}. Por outro lado, o Brasil só possui duas entre as sessenta e cinco publicações retornadas.

Utilizando-se os critérios de inclusão e exclusão descritos nas seções \ref{section:criterios_inclusao} e \ref{section:criterios_exclusao}, respectivamente, dos 65 artigos retornados, 27 foram selecionados inicialmente. Após realizada as leituras, apenas 16 publicações mostraram-se relevantes para responder às questões de pesquisa e/ou apoiar na elaboração do estudo de caso. Esta listagem está disponível no \autoref{quad:publicacoes_selecionadas}. 

\begin{quadro}
\centering

\setlength{\tabcolsep}{0.8em} % for the horizontal padding
\renewcommand{\arraystretch}{1.5}% for the vertical padding
\begin{tabular}{p{3in}|p{1in}|p{0.5}}
\hline

\multicolumn{1}{|p{4in}}{\textbf{Título}} & 
\multicolumn{1}{|p{0.5in}|}{\textbf{Ano}} \\
\hhline{---}

\multicolumn{1}{|p{4in}}{\english{Microservice Migration Using Strangler Fig Pattern and Domain-Driven Design}} & 
\multicolumn{1}{|p{0.5in}|}{\citeyear{ma20221285}} \\
\hhline{---}

\multicolumn{1}{|p{4in}}{\english{Microservice architecture and model-driven development: Yet singles, Soon Married ?}} & 
\multicolumn{1}{|p{0.5in}|}{\citeyear{Rademacher2018}} \\
\hhline{---}

\multicolumn{1}{|p{4in}}{\english{Does Domain-Driven Design Lead to Finding the Optimal Modularity of a Microservice?}} & 
\multicolumn{1}{|p{0.5in}|}{\citeyear{Vural202132721}} \\
\hhline{---}

\multicolumn{1}{|p{4in}}{\english{Modeling Microservices with DDD}} & 
\multicolumn{1}{|p{0.5in}|}{\citeyear{Merson20207}} \\
\hhline{---}

\multicolumn{1}{|p{4in}}{\english{Following Domain Driven Design principles for Microservices decomposition: Is it enough?}} & 
\multicolumn{1}{|p{0.5in}|}{\citeyear{Farsi2021}} \\
\hhline{---}

\multicolumn{1}{|p{4in}}{\english{An Ontology-based Approach for Domain-driven Design of Microservice Architectures}} & 
\multicolumn{1}{|p{0.5in}|}{\citeyear{Diepenbrock20171777}} \\
\hhline{---}

\multicolumn{1}{|p{4in}}{\english{Challenges of domain-driven microservice design: A model-driven perspective}} & 
\multicolumn{1}{|p{0.5in}|}{\citeyear{Rademacher201836}} \\
\hhline{---}

\multicolumn{1}{|p{4in}}{\english{Model-based engineering for microservice architectures using Enterprise Integration }} & 
\multicolumn{1}{|p{0.5in}|}{\citeyear{Petrasch2017}} \\
\hhline{---}

\multicolumn{1}{|p{4in}}{\english{Patterns on Deriving APIs and their Endpoints from Domain Models }} & 
\multicolumn{1}{|p{0.5in}|}{\citeyear{Singjai2021}} \\
\hhline{---}

\multicolumn{1}{|p{4in}}{\english{Refactoring with domain-driven design in an industrial context: An action research report}} & 
\multicolumn{1}{|p{0.5in}|}{\citeyear{Ozkan2023}} \\
\hhline{---}

\multicolumn{1}{|p{4in}}{\english{A microservice based reference architecture model in the context of enterprise architecture}} & 
\multicolumn{1}{|p{0.5in}|}{\citeyear{Yale20171856}} \\
\hhline{---}

\multicolumn{1}{|p{4in}}{\english{Design of Domain-driven Microservices-based Software Talent Evaluation and Recommendation System}} & 
\multicolumn{1}{|p{0.5in}|}{\citeyear{Zhang2022310}} \\
\hhline{---}

\multicolumn{1}{|p{4in}}{\english{Partitioning microservices: A domain engineering approach}} & 
\multicolumn{1}{|p{0.5in}|}{\citeyear{Joselyne201843}} \\
\hhline{---}

\multicolumn{1}{|p{4in}}{\english{Practitioner Views on the Interrelation of Microservice APIs and Domain-Driven Design: A Grey Literature Study Based on Grounded Theory}} & 
\multicolumn{1}{|p{0.5in}|}{\citeyear{Singjai202125}} \\
\hhline{---}

\multicolumn{1}{|p{4in}}{\english{A Systematic Framework of Application Modernization to Microservice based Architecture}} & 
\multicolumn{1}{|p{0.5in}|}{\citeyear{Joselyne2021}} \\
\hhline{---}

\multicolumn{1}{|p{4in}}{\english{Domain-specific language and tools for strategic domain-driven design, context mapping and bounded context modeling}} & 
\multicolumn{1}{|p{0.5in}|}{\citeyear{Kapferer2020299}} \\
\hhline{---}

\end{tabular}
\caption{Publicações selecionadas}
\label{quad:publicacoes_selecionadas}
\end{quadro}


\section{Estratégias para elaboração de sistemas com microsserviços e DDD}
Todas as publicações revisadas sugerem o mapeamento de cada \nameref{section:bounded_context} para um microsserviço. \citeonline{Vural202132721} afirmam que o objetivo principal da delimitação é alcançar serviços com baixo acoplamento e alta coesão. Porém, o grande desafio está na definição de maneira apropriada do escopo de cada \acrshort{bc}.

\citeonline{Singjai202125} e \citeonline{ma20221285} mencionam os padrões \english{\acrfull{ohs}} e \english{\acrfull{acl}} como estratégias para comunicação, conceitualmente, entre \hyperref[section:bounded_context]{Bounded Contexts}. No \acrshort{ohs}, o serviço que envia as mensagens implementa uma camada extra com objetivo de realizar a tradução para um formato que possa ser processado pelo serviço receptor. Dessa forma, os detalhes de implementação do serviço cliente não são expostos, diminuindo o acoplamento entre as partes. Semelhantemente, o \acrshort{acl} é uma camada extra inserida no serviço que recebe as mensagens. Trata-se de uma abordagem útil quando o serviço receptor não deseja aderir ao contrato do componente que produz as mensagens \cite{evans2004ddd}.

Outro ponto importante levantando no material revisado é o mapeamento do modelo de domínio para uma \acrfull{api} para possibilitar o consumo das funcionalidades tanto por outros serviços quanto de aplicações cliente. Algumas opções citadas por \citeonline{Singjai202125} são:
\begin{itemize}
    \item Expor todo o modelo de domínio em uma relação de 1 para 1 com a \acrshort{api}.
    \item Expor uma parte do modelo como uma \acrshort{api}.
    \item Expor cada \acrshort{bc} como uma \acrshort{api}.
    \item Expor parte dos \acrshort{bc}s como \acrshort{api}s.
\end{itemize}

As alternativas mais indicadas são a exposição total ou de grupo de \acrshort{bc}s como \acrshort{api}s. Além disso, \citeonline{Singjai202125} e \citeonline{Singjai2021} levantam abordagens para definição do contrato de \acrshort{api} de acordo com modelo de domínio. As duas principais opões apresentadas pelos autores são: explicitamente definir o contrato de \acrshort{api} e extrair o contrato de \acrshort{api} a partir do modelo. \citeonline{Singjai202125} argumenta que ambas opções auxiliam na obtenção da separação do contrato de \acrshort{api} e das responsabilidades do modelo de domínio.

Para este trabalho de conclusão de curso, a estratégia de mapeamento de um \acrshort{bc} para um microsserviço é utilizada. Com objetivo de habilitar a comunicação entre subdomínios, o padrão \acrfull{ohs} é empregado. Adicionalmente, o contrato de \acrshort{api} será definido explicitamente a partir do modelo de domínio. Assim, pretende-se utilizar as estratégias mais indicadas pelos autores revisados na elaboração do estudo de caso.

\section{Desafios na utilização de DDD como estratégia de delimitação de microsserviços}
\citeonline{Rademacher201836} menciona três principais desafios da utilização de \acrshort{ddd} no contexto da \acrshort{ams}. Adicionalmente, \citeonline{Diepenbrock20171777} cita desafios semânticos juntamente com sua proposta de metamodelo.

A ciência desse conjunto de desafios é crucial para elaboração do estudo de caso deste trabalho de conclusão de curso.

\subsection{Extraindo microsserviços a partir de modelos de domínio}
Para fomentar o foco em conceitos relevantes e \english{design} efetivo, modelos de domínio tipicamente omitem informações obrigatórias para extração de microsserviços, como:
\begin{itemize}
    \item Interfaces e operações;
    \item Parâmetros e tipos de retorno das operações
    \item \english{Endpoints}, protocolos e formatos de mensagens.
\end{itemize}
\cite{Rademacher201836}.

Informações específicas sobre esses pontos são cruciais para implementação dos serviços. Considerando diferentes \acrshort{bc}s conectados entre si no modelo, e que cada \acrshort{bc} é mapeado para um microsserviço, um componente vai necessitar acessar instâncias de um outro serviço. Assim, uma operação desse tipo deve ser fornecida e usualmente não é especificada no modelo, deixando espaço para ambiguidade \cite{Rademacher201836}.

\subsection{Componentes de infraestrutura faltantes no modelo de domínio}
Modelos de domínio intencionalmente não compreendem componentes de infraestrutura da \acrshort{ams} \cite{Rademacher201836}. Esses componentes incluem \english{API Gateways}, \english{containers}, banco de dados, entre outros. Por outro lado, requisitos do modelo podem afetar questões técnicas e esses pontos devem ser documentados separadamente do modelo de domínio \cite{Rademacher201836}. Por exemplo, caso um microsserviço que realize o gerenciamento de usuários necessite ser acessível externamente para faturamento, configurações em diferentes componentes técnicos como \english{API Gateway} necessitarão ser aplicadas.

\subsection{Modelagem de domínio autônoma}
Responsabilidade sobre um microsserviço é geralmente atribuída a um único time, devido à alta coesão e baixo acoplamento \cite{Rademacher201836}. Essa equipe é responsável pela implementação do serviço, operação, \english{design} e manutenção desse \acrfull{bc}. Dessa forma, surgem desafios relacionados a visibilidade dos modelos e gerenciamento de alterações.

Primeiramente, é crucial definir a visibilidade de cada \acrshort{bc}, ou seja, especificar quais equipes terão acesso a quais modelos de domínios \cite{Rademacher201836}. Além disso, a permissão para realizar alterações é uma consideração essencial. Embora seja possível conceder a outras equipes privilégios para modificar outros \acrshort{bc}, essa abordagem apresenta a desvantagem de possibilitar alterações, por vezes críticas, efetuadas por profissionais que não estão familiarizados com o contexto específico. No entanto, restringir exclusivamente à equipe responsável a capacidade de realizar mudanças pode resultar em gargalos significativos, especialmente em projetos envolvendo diversos contextos.

\subsection{Desafios semânticos}
\citeonline{Diepenbrock20171777} apresenta uma série de desafios semânticos na elaboração de microsserviços com \acrshort{ddd}. Inicialmente, um problema de semântico típico ocorre quando um atributo de um conceito de domínio é derivado de outro atributo. Por exemplo, quando um atributo de uma entidade é criado a partir da concatenação de dois outros atributos de outra entidade, ocorre um problema de semântica porque os dados ficam fragmentados.

Simultaneamente, é importante reconhecer que diferentes \acrshort{bc}s podem interpretar os conceitos de domínio compartilhados de maneiras distintas \cite{Diepenbrock20171777}. Nesse contexto, surgem desafios como atributos com nomes diversos, mas significados idênticos, bem como propriedades com o mesmo nome, porém, com significados diferentes. Esse risco é amplificado no contexto da \acrshort{ams}, onde é comum que diferentes \acrshort{bc}s sejam desenvolvidos por equipes distintas. Esse ambiente descentralizado pode potencializar a disparidade de interpretações e a falta de consistência nos conceitos de modelos compartilhados.

Além disso, um mecanismo de definição de identificadores únicos deve ser definido entre as equipes para permitir a identificação semântica de diferentes conceitos do modelo com objetivo de tornar os serviços capazes de distinguir diferentes elementos em um contexto distribuído \cite{Diepenbrock20171777}.

Por essas motivações, \citeonline{Diepenbrock20171777} apresenta uma abordagem para \acrshort{ddd} no contexto da \acrshort{ams}. O autor propõe um metamodelo que representa a sintaxe abstrata de um linguagem formal de modelagem. Em outras palavras, ele define os conceitos suportados pela linguagem e seus relacionamentos. Os principais componentes do metamodelo são: \english{External Context}, \english{\acrlong{bc}}, \english{Domain Model}, \english{Attribute} e \english{Association}. O objetivo principal desse modelo é minimizar os impactos dos desafios semânticos da utilização dessas tecnologias \cite{Diepenbrock20171777}.

\section{Anti-padrões a serem evitados}
\citeonline{Farsi2021}, \citeonline{Vural202132721}, \citeonline{Singjai2021}, \citeonline{Ozkan2023} e \citeonline{Singjai202125} identificaram uma série de anti-padrões no contexto da \acrshort{ams} e \acrshort{ddd}. Um compilado pode ser observado no \autoref{quad:anti_padroes}.

\begin{quadro}
\centering
\caption{Anti-padrões}
\setlength{\tabcolsep}{0.8em} % for the horizontal padding
\renewcommand{\arraystretch}{1.5}% for the vertical padding
\begin{tabular}{|p{1in}|p{3.7in}|}
\hline

\textbf{Nome} & \textbf{Descrição} \\ \hline
\english{Chattiness of a service} &  Refere-se a excessiva comunicação entre microsserviços que gera ineficiência devido à latência de rede. \\ \hline
\english{Nanoservice} & Um serviço excessivamente granular no qual a sobrecarga de comunicação, manutenção e operação supera sua utilidade. \\ \hline
\english{Anemic domain model} & Trata-se de um modelo de domínio em que os objetos contém pouca ou nenhuma regra de negócio. As invariantes são misturadas com outras lógicas, o que dificulta manutenção e refatoração. \\ \hline
\english{Data class} & Similar ao \english{Anemic domain model}, esse tipo de classe só contem, além de atributos, \english{getters} e \english{setters}. Os comportamentos são criados fora da classe, o que reduz coesão e dificulta a manutenção. \\ \hline
\english{Distributed monoliths} & Refere-se a um sistema que externamente se assemelha a \acrlong{ams}, porém possui um alto nível de acoplamento entre os componentes, diminuindo assim as vantagens dos microsserviços. \\ \hline
\english{Start with API Design} & Trata-se de uma abordagem na qual o contrato de \acrshort{api} é definido antes do modelo de domínio. \citeonline{Singjai2021} levantou que esta estratégia costuma gerar a criação de \english{Anemic domain models}. \\ \hline
\english{Feature envy} & Acontece quando um método está mais interessado em uma classe diferente da que está inserido. \\ \hline
\english{Inappropiate intimacy} & Descreve um par de classes não relacionadas conceitualmente, mas que possuem grande acoplamento entre si. \\ \hline
\english{Message chain} & Uma cadeia de mensagem ocorre quando um cliente envia uma mensagem a outro objeto que, por sua vez, a envia outro o objeto e assim por diante. \\ \hline

\end{tabular}
\fonte{\citeonline{Farsi2021}, \citeonline{Vural202132721}, \citeonline{Singjai2021}, \citeonline{Ozkan2023} e \citeonline{Singjai202125}}
\label{quad:anti_padroes}
\end{quadro}

Os anti-padrões mencionados são importantes para o desenvolvimento do estudo de caso, pois devem ser evitados para que os benefícios da utilização conjunta de \acrshort{ddd} e \acrshort{ams} sejam alcançados.

\section{Discussões}
O mapeamento da literatura mostrou-se efetivo para responder às questões de pesquisa, na medida em que foram encontradas publicações que auxiliam na resolução das perguntas-chave. O levantamento das estratégias, desafios e anti-padrões foi realizado com êxito a partir de artigos de alta qualidade.

No entanto, publicações em outros idiomas que não foram revisadas, assim como publicações não indexadas pelas ferramentas de busca utilizadas, representam uma ameaça a validade da pesquisa. Isso se deve ao fato de que informações essenciais podem não ter sido revisadas devido a essas limitações

Por fim, as informações obtidas nesse mapeamento contribuem tanto para atingir os objetivos gerais da pesquisa quanto para a elaboração do estudo de caso.


\chapter{Metodologia}
\label{cap:metodologia}
Este capítulo apresenta a metodologia e os recursos utilizados para atingir o objetivo do trabalho.

\section{Visão geral}
\begin{figure}[h]
    \centering
    \caption{Etapas de desenvolvimento da pesquisa}
    \includegraphics[width=0.9\textwidth]{media/bpmn_metodo_recurso.png}
    \legend{Fonte: o autor}
    \label{fig:metodo_recurso}
\end{figure}

Pode-se observar na \autoref{fig:metodo_recurso} as etapas de execução dessa pesquisa. Inicialmente, o escopo é definido e o primeiro capítulo é elaborado. Em seguida, a fundamentação teórica com os conceitos-chave é construída. Posteriormente, se realiza um mapeamento da literatura buscando trabalhos similares. Paralelamente a isso, um estudo de caso realista com a utilização de microsserviços e diversos conceitos do \acrshort{ddd} é desenvolvido e um relatório é produzido. Por fim, é escrita a conclusão do trabalho.

\section{Mapeamento sistemático da Literatura}
Esta seção apresenta o protocolo de mapeamento sistemático da literatura usado para atingir os objetivos da pesquisa.

\subsection{Questões de pesquisa}
\label{section:questoes_pesquisa}
\begin{enumerate}
    \item[Q1:] Quais são as principais estratégias para elaboração de sistemas com microsserviços e \acrshort{ddd}?
    \item[Q2:] Quais são os principais desafios na utilização de \acrshort{ddd} como estratégia de delimitação de microsserviços?
    \item[Q2:] Quais são os anti-padrões a serem evitados na utilização de \acrshort{ddd} com microsserviços?
    
\end{enumerate}

\subsection{Estratégia de busca}
Esta seção apresenta a estratégia de buscas de artigos científicos e livros relacionados à pesquisa. As ferramentas utilizadas para realizar as buscas são:
\begin{itemize}
    \item \textbf{Periódicos Capes:} É uma ferramenta disponibilizada pelo governo federal para uso de estudantes e pesquisadores. Acessando através da instituição de ensino ou pesquisa, é possível ter acesso completo a uma grande quantidade de artigos científicos publicados em variadas revistas, conferências e universidades. A principal vantagem dessa ferramenta é a possibilidade de ler o conteúdo integral de grande parte das publicações disponíveis. Por outro lado, as expressões de busca atualmente suportadas são bem limitadas.
    \item \textbf{\english{Scopus:}} Trata-se de um ferramenta similar ao Periódicos Capes. No entanto, o \english{Scopus} permite a elaboração de expressões de buscas mais complexas e sofisticadas, servindo para descobrir publicações não detectadas pelas outras plataformas. Além disso, possui um acervo bem mais amplo que o Periódicos Capes. Entretanto, algumas publicações não podem ser vistas na íntegra de forma gratuita.
    \item \textbf{\english{Google Docs:}} Ferramenta desenvolvida pela \english{Google LLC} que permite a criação e edição de documentos de texto. Suas grandes vantagens em relação a ferramentas de outros fornecedores são as avançadas ferramentas de colaboração e a possibilidade de acesso por meio de navegadores \english{web}, sem necessidade de instalação de \english{software} específico.
    \item \textbf{\english{Google Sheets}}: Com as mesmas características e vantagens do \english{Google Docs}, essa ferramenta fornece recursos para elaboração de planilhas de cálculo. É muito útil para realizar análise de dados simples e também visualizar e apresentar dados tabulares.
\end{itemize}

Como grande parte das publicações na área de computação são em inglês, esta pesquisa utiliza esse idioma para fazer buscas nas ferramentas indicadas. Além disso, \acrfull{ams} e \acrfull{ddd} são relativamente recentes, as buscas se limitaram a publicações feitas nos últimos 20 anos.

Os termos-chave para realização das buscas são: Microsserviço, \acrshort{ddd} e \acrlong{ddd}. Como a busca é feita em inglês, se usará \english{microservice} nas buscas.
\subsection{Expressão de busca}
\label{section:string_busca}

\begin{quadro}[H]
\centering

\setlength{\tabcolsep}{0.8em} % for the horizontal padding
\renewcommand{\arraystretch}{1.5}% for the vertical padding
\caption{Expressão de busca utilizada}
\begin{tabular}{|p{4.5in}|}

\hline
Expressão de Busca \\ \hline
\english{( ( TITLE-ABS-KEY ( microservice ) AND TITLE-ABS-KEY ( domain-driven AND design ) ) OR ( TITLE-ABS-KEY ( microservice ) AND TITLE-ABS-KEY ( ddd ) ) )} \\ \hline

\end{tabular}
\label{quad:string_busca}
\fonte{o autor}
\end{quadro}

No \autoref{quad:string_busca}, percebe-se que a expressão de busca pretende retornar todas as publicações que contenham as palavras chaves no título, resumo ou na seção de \english{keywords}.

\subsection{Estratégia de seleção}
A seguir são apresentados critérios para inclusão de publicações na pesquisa.
\label{section:criterios_inclusao}
\begin{itemize}
    \item Texto completo disponível de forma gratuita pelo portal Periódicos Capes.
    \item Materiais relacionados ao tópico de interesse, ou seja, título ou resumo.
    \item Publicações com ao menos 5 citações.
\end{itemize}

Por outro lado, estes são os critérios para exclusão de publicações.
\label{section:criterios_exclusao}
\begin{itemize}
    \item Publicações duplicadas.
    \item Materiais que não dispõem de informação relevante para responder às questões de pesquisa.
\end{itemize}

\subsection{Estratégia para extração de dados e análise}
Para atingir o objetivo do mapeamento da literatura, são filtrados manualmente nos artigos selecionados segundo os critérios de inclusão e exclusão. A partir da listagem reduzida, todas as publicações são lidas de forma integral.
Adicionalmente, todos os gráficos e tabelas nos artigos selecionados são avaliados visando extrair algum dado que permita realizar a comparação entre aspectos quantitativos das estratégias como, tempo de resposta, latência e taxa de transferência.

Informações qualitativas como recomendações, destaques, conceitos e estudos de casos são registrados em um documento no \english{\href{https://docs.google.com/document/d/1-dXE9_2-CtfDePG9Opkcyipq0F0dcYJ-Z_gbhi_qDhs/edit?usp=sharing}{Google Docs}}. Dados quantitativos como taxa de transferência, latência, tempo de processamento são armazenados em uma planilha no \english{\href{https://docs.google.com/spreadsheets/d/1R-PbCisie8QHARzF2rYtDx8UPOWksEeH-6-SEqIOTLg/edit?usp=sharing}{Google Sheets}}. 

A pesquisa é executada seguindo os passos a seguir:
\begin{enumerate}
    \item As expressões de busca citadas são inseridas nas ferramentas mencionadas.
    \item É feito o armazenamento das publicações retornadas em uma \href{https://docs.google.com/spreadsheets/d/1rtH8Jl1EHguqZ4Py2mgV3pab7IQzt72-Sv2S1jPzLsQ/edit?usp=sharing}{planilha de cálculo}.
    \item As publicações retornadas são filtradas conforme os critérios de inclusão e exclusão.
    \item Em cada artigo selecionado, é realizada a extração dos dados relevantes para responder às questões de pesquisa.
    \item Finalmente, são produzidas respostas para questões de pequisa com as informações extraídas das publicações.
\end{enumerate}

\section{Estudo de Caso}

\subsection{Contexto}
\label{section:contexto}
O estudo de caso é realizado em uma empresa fictícia chamada \emph{CarroFacil}. Essa empresa é uma locadora de veículos que atua em todo o território nacional. Ela possui uma frota de veículos própria e uma grande quantidade de clientes, sendo pessoas físicas ou jurídicas. Eles podem alugar veículos por períodos de tempo variados, desde horas até semanas. Os veículos são retirados em uma das lojas da empresa. A \emph{CarroFacil} possui um sistema de locação de veículos que foi desenvolvido há alguns anos e está apresentando problemas de escalabilidade, desempenho e manutenibilidade. Por esse motivo, a empresa decidiu desenvolver um novo sistema de locação de veículos utilizando microsserviços e \acrshort{ddd}.

\begin{figure}[!h]
    \centering
    \caption{Diagrama de caso de uso}
    \includegraphics[width=0.9\textwidth]{media/diagrama_usecase.png}
    \legend{Fonte: o autor}
    \label{fig:caso_uso}
\end{figure}

A \autoref{fig:caso_uso} apresenta o diagrama de caso de uso do sistema de locação de veículos. As seções a seguir apresentam uma especificação parcial dos requsitos que será aprimorada e detalhada ao decorrer do desenvolvimento da aplicação.

\subsection{Processo de Desenvolvimento}
\label{section:processo_desenvolvimento}
O processo de desenvolvimento utilizado para construir o sistema de locação de veículos é o \english{Kanban}. Essa metodologia de desenvolvimento ágil é baseada em um quadro de tarefas, no qual cada tarefa é representada por um cartão \cite{gomes2014agile}. O quadro é dividido em colunas que representam o estado atual de cada tarefa. As colunas mais comuns são: \english{To Do}, \english{Doing} e \english{Done}. O quadro é atualizado conforme as tarefas são realizadas. A \autoref{fig:kanban} apresenta um exemplo de quadro \english{Kanban} que é utilizado para mapear as tarefas e medir o progresso do desenvolvimento deste sistema.

\begin{figure}[h]
    \centering
    \caption{Exemplo de quadro \english{Kanban}}
    \includegraphics[width=0.9\textwidth]{media/kanban.png}
    \legend{Fonte: o autor}
    \label{fig:kanban}
\end{figure}

\subsection{Requisitos}
A seguir são apresentados os requisitos do sistema de locação de veículos.
\subsubsection{Requisitos Funcionais}

\begin{quadro}[H]
    \centering
    \caption{Registrar cliente}
    \label{quad:registrar_cliente}
    \begin{tabular}{|p{1.2in}|p{3.5in}|}
    \hline
    
    \textbf{Caso de uso} & Registrar cliente \\ \hline
    \textbf{Descrição} & Permite que o cliente se cadastre na plataforma. \\ \hline
    \textbf{Ator} & Cliente \\ \hline
    \textbf{Pre-condições} & O cliente ainda não está cadastrado. \\ \hline
    \textbf{Pós-condições} & O cliente é registrado na plataforma\\ \hline
    \textbf{Cenário Principal} & \begin{enumerate}
        \item O cliente informa o nome, CPF ou CNPJ, e-mail e senha.
        \item O cliente é redirecionado para a tela principal.
    \end{enumerate}  \\ \hline
    \textbf{Fluxo de Exceção} & \begin{enumerate}
        \item Dados inválidos são informados. Nesse caso, o sistema exibe uma mensagem de erro.
        \item Usuário já cadastrado. O sistema também exibe uma mensagem de erro customizada para esse caso.
    \end{enumerate}  \\ \hline
    \end{tabular}
    \fonte{o autor.}
\end{quadro}

O caso de uso do \autoref{quad:registrar_cliente} especifica o registro de um cliente. Esse é o primeiro passo para que um cliente possa realizar uma reserva de veículo.

\begin{quadro}[H]
    \centering
    \caption{Reservar login}
    \label{quad:realizar_login}
    \begin{tabular}{|p{1.2in}|p{3.5in}|}
    \hline
    
    \textbf{Caso de uso} & Realizar login\\ \hline
    \textbf{Descrição} & Permite que um cliente ou funcionário realizem login na plataforma. \\ \hline
    \textbf{Ator} & Cliente \\ \hline
    \textbf{Pre-condições} & O usuário está registrado na plataforma\\ \hline
    \textbf{Pós-condições} & O usuário está autenticado\\ \hline
    \textbf{Cenário Principal} & \begin{enumerate}
        \item Na tela de login, o usuário informa seu email e senha.
        \item O usuário aperta o botão de login.
        \item O usuário é redirecionado para a tela principal.
    \end{enumerate}  \\ \hline
    \textbf{Fluxo de Exceção} & \begin{enumerate}
        \item O email ou senha informados estão incorretos. Nesse caso, o sistema exibe uma mensagem de erro.
    \end{enumerate}  \\ \hline
    \end{tabular} 
    \fonte{o autor.}
\end{quadro}

O caso de uso do \autoref{quad:realizar_login} especifica o processo de autenticação de um usuário. Esse é o segundo passo para que um cliente possa realizar uma reserva de veículo.

\begin{quadro}[H]
    \centering
    \caption{Listar veículos}
    \label{quad:listar_veiculos}
    \begin{tabular}{|p{1.2in}|p{3.5in}|}
    \hline
    
    \textbf{Caso de uso} & Listar veículos \\ \hline
    \textbf{Descrição} & Permite que o cliente visualize os veículos disponíveis para aluguel. \\ \hline
    \textbf{Ator} & Cliente \\ \hline
    \textbf{Pre-condições} & - \\ \hline
    \textbf{Pós-condições} & Uma listagem de veículos disponíveis é exibida. \\ \hline
    \textbf{Cenário Principal} & O cliente acessa a página de veículos, onde é possível visualizar os veículos disponíveis. \\ \hline
    
    \end{tabular}
    \fonte{o autor.}
\end{quadro}

O \autoref{quad:listar_veiculos} especifica o processo de listagem de veículos. Esse processo é realizado pelo cliente e permite que ele visualize os veículos disponíveis para aluguel.

\begin{quadro}[H]
    \centering
    \caption{Reservar um veículo}
    \label{quad:reservar_veiculo}
    \begin{tabular}{|p{1.2in}|p{3.5in}|}
    \hline
    
    \textbf{Caso de uso} & Reservar um veículo \\ \hline
    \textbf{Descrição} & Permite que um cliente reserve um veículo por um período de tempo. \\ \hline
    \textbf{Ator} & Cliente \\ \hline
    \textbf{Pre-condições} & Cliente está autenticado na plataforma. \\ \hline
    \textbf{Pós-condições} & Veículo reservado fica indisponível para outros clientes. \\ \hline
    \textbf{Cenário Principal} & \begin{enumerate}
        \item Na tela de reservas, o usuário seleciona a localização que vai retirar o veículo, o período do aluguel, bem como outras informações.
        \item É exibida uma confirmação para o usuário.
    \end{enumerate}  \\ \hline
    \textbf{Fluxo de Exceção} & \begin{enumerate}
        \item O veículo não está disponível naquela localidade.
        \item O cliente seleciona um período de tempo indisponível.  
        \item Em ambos os casos, o cliente é alertado sobre o problema.
    \end{enumerate}  \\ \hline
    \end{tabular} 
    \fonte{o autor.}
\end{quadro}

O \autoref{quad:reservar_veiculo} especifica o processo de reserva de um veículo. Esse processo é realizado pelo cliente e permite que ele alugue um veículo por um período de tempo.

\begin{quadro}[H]
    \centering
    \caption{Realizar \english{check-in}}
    \label{quad:realizar_checkin}
    \begin{tabular}{|p{1.2in}|p{3.5in}|}
    \hline
    
    \textbf{Caso de uso} & Realizar \english{check-in} \\ \hline
    \textbf{Descrição} & Permite a realização do check-in da reserva entre o cliente e o funcionário. \\ \hline
    \textbf{Ator} & Funcionário \\ \hline
    \textbf{Pre-condições} & \begin{enumerate}
        \item O cliente possui uma reserva e seu identificador.
        \item O horário do \english{check-in} próximo do período da reserva.
    \end{enumerate} \\ \hline
    \textbf{Pós-condições} & O status da reserva é alterado para em progresso. \\ \hline
    \textbf{Cenário Principal} & O cliente apresenta o identificador da reserva ao funcionário que o insere no sistema para realizar o check-in. \\ \hline
    \textbf{Fluxo de Exceção} & \begin{enumerate}
        \item O check-in para aquela reserva já foi realizado.
        \item O cliente chegou à localidade fora do período de \english{check-in}.
        \item Em ambos os casos, o cliente é alertado sobre o problema.
    \end{enumerate}  \\ \hline
    \end{tabular}
    \fonte{o autor.}
\end{quadro}

O caso de uso Realizar \english{check-in} é apresentado no \autoref{quad:realizar_checkin}. Esse caso de uso é realizado pelo funcionário e permite que o cliente retire o veículo reservado.

\begin{quadro}[H]
    \centering
    \caption{Realizar \english{check-out}}
    \label{quad:realizar_checkout}
    \begin{tabular}{|p{1.2in}|p{3.5in}|}
    \hline
    
    \textbf{Caso de uso} & Realizar \english{check-out} \\ \hline
    \textbf{Descrição} & Permite a realização do check-out da reserva entre o cliente e o funcionário. \\ \hline
    \textbf{Ator} & Funcionário \\ \hline
    \textbf{Pre-condições} & \begin{enumerate}
        \item O cliente possui uma reserva em progresso.
        \item O horário atual está próximo do período de \english{check-out} da reserva.
    \end{enumerate} \\ \hline
    \textbf{Pós-condições} & O status da reserva é alterado para concluída. \\ \hline
    \textbf{Cenário Principal} & O cliente retorna o veículo para a loja, o funcionário verifica o estado do veículo, e obtém informações sobre a reserva a partir da placa. Por fim, o funcionário finaliza a reserva. \\ \hline
    \textbf{Fluxo de Exceção} & \begin{enumerate}
        \item O \english{check-in} não foi realizado para reserva.
        \item O cliente chegou à localidade fora do período de \english{check-out}.
        \item Em ambos os casos, o cliente é alertado sobre o problema.
    \end{enumerate} \\ \hline
    \end{tabular}
    \fonte{o autor.}
\end{quadro}

O \autoref{quad:realizar_checkout} especifica o processo de realização do \english{check-out}. Esse processo é realizado pelo funcionário e permite que o cliente devolva o veículo alugado.

\subsubsection{Requisitos não Funcionais}
Com base na quantidade de clientes atuais e a estimativa de crescimento, bem como suas expectativas quanto ao desempenho do sistema, são definidos os seguintes requisitos não funcionais:
\begin{enumerate}
    \item O sistema deve garantir a segurança dos dados dos clientes.
    \item O tempo de resposta do sistema deve ser menor que 500 milissegundos em 99\% das requisições.
    \item O sistema deve suportar 1000 requisições por segundo.
    \item Picos de até 10.000 requisições por segundo devem ser suportados sem degradação do desempenho.
    \item As funcionalidades do sistema devem ser fornecidadas através de uma \acrshort{api} \english{REST}.
\end{enumerate}

\subsection{Design}
Para construção de cada microsserviços, é feito uso da \hyperref[section:hexagonal]{Arquitetura Hexagonal}. Essa, por sua vez, separa a aplicação em duas partes principais: núcleo da aplicação e adaptadores. O núcleo da aplicação contém as regras de negócio e os adaptadores são responsáveis por adaptar a aplicação para o mundo externo. Para modelagem do domínio, é utilizado o \english{\acrfull{ddd}}. O \acrshort{ddd} fornece uma estratégia para modelagem do domínio de negócio, diversos padrões para resolver problemas de modelagem recorrentes e facilidade de entendimento e manutenção de código \cite{evans2004ddd}.

Para realização do \english{design} deste projeto, é utilizado a \english{UML}. Especificamente, são criados diagramas de classes com propósito de modelar os tipos de objeto, seus relacionamentos e os serviços que fornecem. Adicionalmente, para detalhamento do fluxo de execução de operações chaves, alguns diagramas de sequência são empregados. Além disso, visando fornecer uma visão geral da arquiteura do sistema, o diagrama de contexto de sistema do modelo C4 é utilizado \cite{c4Model}. 

\subsection{Implementação}
A implementação desse projeto é feita utilizando a linguagem de programação \english{Java} na versão 17 \cite{java}. Além disso, o \english{framework} \english{Spring Boot} é empregado na construção dos microsserviços Essa ferramenta fornece uma série de recursos para construção de microsserviços, como: injeção de dependências, configuração de banco de dados, configuração de \english{logs}, entre outros \cite{springBoot}. 

São utilizados dois banco de dados para armazenamento dos dados do sistema. O primeiro é um banco de dados relacional, que armazena dados de reservas. O segundo é um banco de dados não relacional, que armazena dados de clientes, veículos, entre outros. Para o banco de dados relacional, é utilizado o \english{PostgreSQL} \cite{postgreSql}. Para o banco de dados não relacional, foi escolhido o \english{Amazon DynamoDB} \cite{dynamoDb}.

O \english{PostgreSQL} é um sistema de gerenciamento de banco de dados relacional de código aberto. Esse banco de dados é um dos mais populares do mundo, sendo utilizado por diversas empresas, como: \english{Apple}, \english{Spotify}, \english{Netflix}, entre outras \cite{postgreSql}. Trata-se de um banco de dados escalável, confiável, fácil de usar e com suporte a transações \acrshort{acid}.

Por outro lado, o \english{Amazon DynamoDB} é um banco de dados não relacional, que fornece desempenho de milissegundos de um dígito a qualquer escala. Esse banco de dados é totalmente gerenciado, ou seja, não é necessário realizar a configuração de servidores, escalabilidade, replicação, entre outros. Além disso, o \english{Amazon DynamoDB} é um banco de dados sem servidor, ou seja, o usuário paga apenas pelo que utiliza \cite{dynamoDb}.

Para a intercomunicação entre os microsserviços de maneira assíncrona, é utilizado o \english{Amazon SQS}. Esse \english{broker} de mensagens permite que os microsserviços se comuniquem de maneira assíncrona, sem que um microsserviço precise conhecer o outro \cite{amazonSqs}. Além disso, o \english{Amazon SQS} permite que as mensagens sejam armazenadas em uma fila, caso o microsserviço destinatário da mensagem esteja fora do ar.

\subsection{Validação}
O projeto é desenvolvido com as estratégias \acrfull{tdd} e \acrfull{bdd}. Essas abordagens de desenvolvimento de \english{software} permitem que o código seja desenvolvido de maneira mais confiável e com maior qualidade \cite{barauna2020tdd}. Além disso, o \acrshort{tdd} e o \acrshort{bdd} permitem que o código seja desenvolvido de maneira mais rápida, pois os testes são escritos antes do código.

Três tipos de testes são escritos para validar os requisitos funcionais do sistema: testes unitários, testes de integração e testes de aceitação. Os testes unitários são escritos para validar métodos e classes do sistema. Por outro lado, os testes de integração validam a integração entre dois microsserviços e integrações com componentes externos, como banco de dados. Por fim, os testes de aceitação são implementados para validar os requisitos do sistema. É importante ressaltar que todos os testes são automatizados utilizando o \english{JUnit} \cite{junit}.

Por outro lado, para validação dos requisitos não funcionais, testes de carga são empregados. O ambiente de execução desses testes é o mesmo de implantação, definido na seção \ref{section:implantacao}. Através da ferramenta Gatling \cite{gatling}, a bateria de testes consiste de 1000 operações sendo executadas a cada minuto por um período de 1 hora. Além disso, a cada 15 minutos há um pico de 10000 operações em um minuto. São avaliados o tempo de resposta, a utilização da CPU, o consumo de memória e a quantidade erros. Os resultados obtidos são comparados com o \autoref{quad:metricas_comparacao}. Cada operação possui as seguintes etapas: registrar um novo veículo, cadastrar um cliente, listar veículos e realizar uma reserva.

Após a execução dos testes de carga, os resultados obtidos são analizados e comparados com os requisitos não funcionais. Dessa forma, um relatório é produzido com uma análise gráfica e númerica.

\begin{quadro}[H]
    \centering
    \caption{Métricas de comparação}
    \label{quad:metricas_comparacao}
    \begin{tabular}{|p{1.2in}|p{3.5in}|}
    \hline
    
    \textbf{Métrica} & \textbf{Alvo} \\ \hline
    Tempo de resposta & <= 500 ms em 99\% das requisições. \\ \hline
    Utilização da CPU & <= 70\% durante toda a execução. \\ \hline
    Consumo de memória & <= 70\% durante toda a execução. \\ \hline
    Quantidade de erros & <= 1\% das requisições. \\ \hline

    \end{tabular}
    \fonte{o autor.}
\end{quadro}

Os parâmetros descritos no \autoref{quad:metricas_comparacao} foram definidos pelo autor com base em experiências anteriores de desenvolvimento de sistemas similares no ambiente corporativo, juntamente com a recomendação do orientador.

\subsection{Implantação}
\label{section:implantacao}
O sistema é implantado na nuvem da \english{\acrfull{aws}}. A \acrshort{aws} é uma plataforma de computação em nuvem que oferece mais de 200 serviços completos de \english{data centers} em todo o mundo \cite{aws}. Esses serviços incluem computação, armazenamento, banco de dados, \english{networking}, \english{analytics}, \english{machine learning}, inteligência artificial, \english{Internet of Things}, segurança, entre outros.

Cada microsserviço é implantado em um \english{container} \english{Docker}. Essa tecnologia permite que os microsserviços sejam executados de maneira isolada, sem que um microsserviço interfira no outro. Além disso, o \english{Docker} permite que os microsserviços sejam executados em qualquer ambiente, sem que seja necessário realizar alterações no código \cite{docker}. O serviço \english{Amazon ECS} é utilizado para orquestrar os \english{containers} \english{Docker}. Esse serviço permite que os \english{containers} sejam executados em um ambiente de produção de maneira escalável e confiável \cite{amazonEcs}.

Um \english{cluster} \english{PostgreSQL} com o auxílio do serviço \english{Amazon RDS} é utilizado \cite{postgreSql}. Por outro lado, o \english{Amazon DynamoDB} e o \english{Amazon SQS} são soluções nativas da \acrshort{aws}.

Para a orquestração e automatização do provisionamento e configuração da infraestrutura, é feito uso da ferramenta \english{Terraform}. Essa ferramenta permite que a infraestrutura seja definida como código, ou seja, é possível definir a infraestrutura utilizando uma linguagem de programação \cite{terraform}. Além disso, o \english{Terraform} permite que a infraestrutura seja provisionada e configurada de maneira automatizada.


\chapter{Cronograma}
A seguir será apresentado o cronograma de atividades para o desenvolvimento deste trabalho de conclusão de curso.
\begin{quadro}[H]
    \centering
    \caption{Cronograma de Atividades}
    \begin{tabular}{|p{3.5cm}|c|c|c|c|c|c|c|c|c|c|c|c|}
        \hline
        \multirow{2}{*}{\textbf{Atividades}} & \multicolumn{4}{c|}{\textbf{2023}} & \multicolumn{7}{c|}{\textbf{2024}} \\ \cline{2-12}
        & \textbf{09} & \textbf{10} & \textbf{11} & \textbf{12} & \textbf{01} & \textbf{02} & \textbf{03} & \textbf{04} & \textbf{05} & \textbf{06} & \textbf{07} \\ \hline
        Escolha do Tema & x & & & & & & & & & & \\ \hline
        Entender contexto do problema & x & & & & & & & & & & \\ \hline
        Definir Objetivos & & x & & & & & & & & & \\ \hline
        Elaborar Justificativa & & x & & & & & & & & & \\ \hline
        Revisar bibliografia & & x & & & & & & & & & \\ \hline
        Realizar Mapeamento da Literatura & & & x & & & & & & & & \\ \hline
        Definir Requisitos & & & * & & & & & & & & \\ \hline
        Especificar processo de desenvolvimento & & & * & & & & & & & & \\ \hline
        Realizar análise requisitos & & & & & & & & & & & \\ \hline
        Elaborar \textit{Design} da solução & & & & & & & & & & & \\ \hline
        Realizar implementação & & & & & & & & & & & \\ \hline
        Executar testes para validação & & & & & & & & & & & \\ \hline
        Elaborar conclusão & & & & & & & & & & & \\ \hline
        Apresentar TCC & & & & & & & & & & & \\ \hline
      \end{tabular}
    \label{quad:cronograma}
    \fonte{o autor}
\end{quadro}
% \chapter{Desenvolvimento}
\label{cap:desenvolvimento}
Neste capítulo, é apresentado as etapas de desenvolvimento do projeto.

\section{Levantamento das Estórias}

As estórias foram identificadas, analisadas e descritas abaixo:

\begin{itemize}
	\item \textbf{Estória 01: Cadastrar Catador}\par
Descrição: Permite a criação do usuário responsável pelo recolhimento do material descartado.

	\item \textbf{Estória 02: Cadastrar Separador}\par
Descrição: Permite a criação do usuário responsável pelo descarte do material.

	\item \textbf{Estória 03: Efetuar login}\par
Descrição: Autoriza o acesso às funcionalidades do sistema. 

	\item \textbf{Estória 04: Recuperar senha }\par
Descrição: Possibilita que o usuário redefina a senha.

	\item \textbf{Estória 05: Tirar dúvidas }\par
Descrição: Esclarece sobre a definição dos perfis. 

	\item \textbf{Estória 06: Solicitar recolhimento}\par
Descrição: Requisita a coleta do material descartado.

	\item \textbf{Estória 07: Realizar logoff}\par
Descrição: Permite que o usuário saia do sistema.

	\item \textbf{Estória 08: Lembrar login}\par
Descrição: Permite que os dados de autenticação do usuário, fiquem salvos na tela de \textit{login}.

	\item \textbf{Estória 9: Traçar rota da coleta}\par
Descrição: Possibilita que o usuário visualize no mapa a duração e a distância do trajeto a ser percorrido.

	\item \textbf{Estória 10: Visualizar mapa}\par
Descrição: Facilitar a visualização da localização.

	\item \textbf{Estória 11: Escolher tipo de material}\par
Descrição: Informar qual tipo de material reciclável será descartado.

	\item \textbf{Estória 12: Informar localização}\par
Descrição: Informar a localização para o descarte do material.
\end{itemize}


\section{Diagramas do Modelo Proposto}

\subsection{Diagrama de Classes}

Na construção do diagrama de classes foi analisada as principais classes do sistema, suas características, seus relacionamentos e suas funcionalidades. É possível observar através da \autoref{fig:classe}, a estrutura do desenvolvimento do sistema.

%%%%%%%%%%%%%%%%%%%% Figure/Image No: 8 starts here %%%%%%%%%%%%%%%%%%%%

\begin{figure}[H]
	\begin{Center}
		\includegraphics[width=6.4in,height=2.77in]{./media/image35.png}
	\end{Center}
	\caption{Diagrama de Classes}
	\label{fig:classe}
\end{figure}

%%%%%%%%%%%%%%%%%%%% Figure/Image No: 8 Ends here %%%%%%%%%%%%%%%%%%%%

Na \autoref{fig:classe} citada acima podemos observar o Diagrama de Classes utilizado para a construção do aplicativo.

As classes Catador e Separador são extensões da classe Pessoa, que tem como objetivo informar seus dados, tirar dúvidas sobre a definição dos perfis, criar uma conta perfil, redefinir a senha esquecida, efetuar \textit{login}, salvar os dados do \textit{login} e realizar \textit{logoff}.

A classe Catador que faz parte de uma das contas perfil, é responsável pelo cadastro do usuário catador. Este usuário é encarregado por traçar a rota da coleta e recolher o material descartado.

A classe Separador que faz parte de uma das contas perfil, é responsável pelo cadastro do usuário separador. Ao informar sua localização e o tipo de material, o usuário poderá solicitar o recolhimento do material descartado.

\subsection{Diagrama de Casos de Uso}

Esse parágrafo descreve as funcionalidades do sistema e as interações entre os atores, através da \autoref{fig:casos}. 

%%%%%%%%%%%%%%%%%%%% Figure/Image No: 9 starts here %%%%%%%%%%%%%%%%%%%%

\begin{figure}[H]
	\begin{Center}
		\includegraphics[width=5.12in,height=3.02in]{./media/image32.png}
	\end{Center}
	\caption{Diagrama de Casos de Uso}
	\label{fig:casos}
\end{figure}

%%%%%%%%%%%%%%%%%%%% Figure/Image No: 9 Ends here %%%%%%%%%%%%%%%%%%%%

Na \autoref{fig:casos} citada acima, podemos ver o Diagrama de Casos de Uso. Na tela inicial do aplicativo, o usuário poderá realizar as seguintes funções: logar no APP, cadastrar usuário, tirar dúvidas, efetuar \textit{logoff} e recuperar a senha. Se o usuário ainda não estiver cadastrado, irá se cadastrar no perfil desejado: Catador ou Separador. Caso já tenha cadastro, basta efetuar o \textit{login}, o qual pode ser salvo pelo sistema. Em caso de senha esquecida, o sistema permite a recuperação através do \textit{e-mail}.

A partir da inserção da localização e do tipo de material, o Separador poderá solicitar o recolhimento do rejeito. O Catador recebendo a solicitação do Separador, irá traçar a rota para a coleta do material. 

\paragraph*{5.2.2.1 Especificação dos Casos de Uso}

Esse capítulo é responsável pelo detalhamento das funcionalidades do sistema. O (\autoref{quad:catador}), descreve o passo a passo para o cadastro do usuário Catador.

%%%%%%%%%%%%%%%%%%%% Table No: 4 starts here %%%%%%%%%%%%%%%%%%%%

\begin{quadro}[H]
\caption{Cadastrar Catador}
\label{quad:catador}
\centering
\begin{tabular}{p{1.25in}p{4.50in}}
\hline
%row no:1
\multicolumn{1}{|p{1.25in}}{\textbf{Caso de Uso}} & 
\multicolumn{1}{|p{4.50in}|}{Cadastrar Catador} \\
\hhline{--}
%row no:2
\multicolumn{1}{|p{1.25in}}{\textbf{Descrição}} & 
\multicolumn{1}{|p{4.50in}|}{Permite a criação de perfil usuário denominado catador.} \\
\hhline{--}
%row no:3
\multicolumn{1}{|p{1.25in}}{\textbf{Ator}} & 
\multicolumn{1}{|p{4.50in}|}{Catador} \\
\hhline{--}
%row no:4
\multicolumn{1}{|p{1.25in}}{\textbf{Pré-condições}} & 
\multicolumn{1}{|p{4.50in}|}{Preencher todos os dados solicitados} \\
\hhline{--}
%row no:5
\multicolumn{1}{|p{1.25in}}{\textbf{Pós-condições}} & 
\multicolumn{1}{|p{4.50in}|}{Cadastro do perfil catador no sistema} \\
\hhline{--}
%row no:6
\multicolumn{1}{|p{1.25in}}{\textbf{Fluxo Principal}} & 
\multicolumn{1}{|p{4.50in}|}{\begin{enumerate}[label*={\fontsize{12pt}{12pt}\selectfont \arabic*.}]
	\item Na tela inicial, o usuário solicita o cadastro; \par 	\item Em seguida preenche todos os dados; \par 	\item O sistema valida os dados preenchidos; \par 	\item O cadastro é realizado.
\end{enumerate}} \\
\hhline{--}
%row no:7
\multicolumn{1}{|p{1.25in}}{\textbf{Fluxo de Exceção}} & 
\multicolumn{1}{|p{4.50in}|}{\textbf{Dados incorretos} \par \begin{enumerate}[label*={\fontsize{12pt}{12pt}\selectfont \arabic*.}]
	\item No passo 3 do Fluxo Principal, o usuário não preencheu os dados corretamente, o sistema sinaliza qual campo não foi preenchido.
\end{enumerate} \par \textbf{Dados não cadastrados} \par \begin{enumerate}[label*={\fontsize{12pt}{12pt}\selectfont \arabic*.}]
	\item No passo 3 do Fluxo Principal, o usuário preenche algum campo já cadastrado, o sistema exibe qual campo já foi cadastrado.
\end{enumerate}} \\
\hhline{--}

\end{tabular}
\end{quadro}


%%%%%%%%%%%%%%%%%%%% Table No: 4 ends here %%%%%%%%%%%%%%%%%%%%

O caso de uso Cadastrar Catador (\autoref{quad:catador}), citado acima é responsável por avaliar as ações necessárias para realizar o cadastro do perfil catador. 

O \autoref{qua:separador} descreve o passo a passo para o cadastro do usuário Separador.

%%%%%%%%%%%%%%%%%%%% Table No: 5 starts here %%%%%%%%%%%%%%%%%%%%


\begin{quadro}[H]
\caption{Cadastro Separador}
 \label{qua:separador}			\begin{tabular}{p{1.33in}p{3.96in}}
\hline
%row no:1
\multicolumn{1}{|p{1.33in}}{\textbf{Caso de Uso}} & 
\multicolumn{1}{|p{3.96in}|}{Cadastrar Separador} \\
\hhline{--}
%row no:2
\multicolumn{1}{|p{1.33in}}{\textbf{Descrição}} & 
\multicolumn{1}{|p{3.96in}|}{Permite a criação de perfil usuário denominado separador.} \\
\hhline{--}
%row no:3
\multicolumn{1}{|p{1.33in}}{\textbf{Ator}} & 
\multicolumn{1}{|p{3.96in}|}{Separador} \\
\hhline{--}
%row no:4
\multicolumn{1}{|p{1.33in}}{\textbf{Pré-condições}} & 
\multicolumn{1}{|p{3.96in}|}{Preencher todos os dados solicitados } \\
\hhline{--}
%row no:5
\multicolumn{1}{|p{1.33in}}{\textbf{Pós-condições}} & 
\multicolumn{1}{|p{3.96in}|}{Cadastro do perfil separador no sistema} \\
\hhline{--}
%row no:6
\multicolumn{1}{|p{1.33in}}{\textbf{Fluxo Principal}} & 
\multicolumn{1}{|p{3.96in}|}{\begin{enumerate}[label*={\fontsize{12pt}{12pt}\selectfont \arabic*.}]
	\item Na tela inicial, o usuário solicita o cadastro; \par 	\item Em seguida, preenche todos os dados; \par 	\item O sistema verifica os dados; \par 	\item O cadastro é realizado.
\end{enumerate}} \\
\hhline{--}

\end{tabular}
 \end{quadro}

%%%%%%%%%%%%%%%%%%%% Table No: 5 ends here %%%%%%%%%%%%%%%%%%%%

O caso de uso Cadastrar Separador (\autoref{qua:separador}), citado acima é responsável por avaliar as ações necessárias para realizar o cadastro do perfil separador. 

O \autoref{quad:login} descreve o passo a passo para efetuar o \textit{login}.

%%%%%%%%%%%%%%%%%%%% Table No: 6 starts here %%%%%%%%%%%%%%%%%%%%


\begin{quadro}[H]
\caption{Efetuar Login}
\label{quad:login}
\centering
\begin{tabular}{p{1.35in}p{3.94in}}
\hline
%row no:1
\multicolumn{1}{|p{1.35in}}{\textbf{Caso de Uso}} & 
\multicolumn{1}{|p{3.94in}|}{Efetuar \textit{login}} \\
\hhline{--}
%row no:2
\multicolumn{1}{|p{1.35in}}{\textbf{Descrição}} & 
\multicolumn{1}{|p{3.94in}|}{Permite acesso às funcionalidades do sistema.} \\
\hhline{--}
%row no:3
\multicolumn{1}{|p{1.35in}}{\textbf{Ator}} & 
\multicolumn{1}{|p{3.94in}|}{Catador e Separador} \\
\hhline{--}
%row no:4
\multicolumn{1}{|p{1.35in}}{\textbf{Pré-condições}} & 
\multicolumn{1}{|p{3.94in}|}{O usuário deve estar cadastrado no sistema.} \\
\hhline{--}
%row no:5
\multicolumn{1}{|p{1.35in}}{\textbf{Pós-condições}} & 
\multicolumn{1}{|p{3.94in}|}{O usuário estará logado no sistema.} \\
\hhline{--}
%row no:6
\multicolumn{1}{|p{1.35in}}{\textbf{Fluxo Principal}} & 
\multicolumn{1}{|p{3.94in}|}{\begin{enumerate}[label*={\fontsize{12pt}{12pt}\selectfont \arabic*.}]
	\item Na tela inicial o usuário, informa o \textit{e-mail} e senha; \par 	\item Em seguida realiza o \textit{login}; \par 	\item O sistema verifica os dados; \par 	\item O usuário acessa a aplicação.
\end{enumerate}} \\
\hhline{--}
%row no:7
\multicolumn{1}{|p{1.35in}}{\textbf{Fluxo de Exceção}} & 
\multicolumn{1}{|p{3.94in}|}{\textbf{Usuário ou senha inválidos} \par \begin{enumerate}[label*={\fontsize{12pt}{12pt}\selectfont \arabic*.}]
	\item No passo 1, o usuário informa \textit{e-mail} e/ou senha não cadastrados; \par 	\item O sistema informa que os dados estão incorretos.
\end{enumerate} \par \textbf{Campo nulo} \par \begin{enumerate}[label*={\fontsize{12pt}{12pt}\selectfont \arabic*.}]
	\item No passo 1, o usuário não preenche algum campo; \par 	\item O sistema informa qual campo não foi preenchido; \par 	\item Retorna para o passo 1.
\end{enumerate}} \\
\hhline{--}

\end{tabular}
\end{quadro}

%%%%%%%%%%%%%%%%%%%% Table No: 6 ends here %%%%%%%%%%%%%%%%%%%%

O caso de uso Efetuar Login (\autoref{quad:login}), citado acima é responsável por avaliar as ações necessárias para realizar o \textit{login }do usuário. 

O \autoref{quad:recusenha} descreve o passo a passo para a recuperação da senha.

%%%%%%%%%%%%%%%%%%%% Table No: 7 starts here %%%%%%%%%%%%%%%%%%%%


\begin{quadro}[H]
\caption{Recuperar Senha}
\label{quad:recusenha}
\centering
\begin{tabular}{p{1.35in}p{3.84in}}
\hline
%row no:1
\multicolumn{1}{|p{1.35in}}{\textbf{Caso de Uso}} & 
\multicolumn{1}{|p{3.84in}|}{Recuperar senha} \\
\hhline{--}
%row no:2
\multicolumn{1}{|p{1.35in}}{\textbf{Descrição}} & 
\multicolumn{1}{|p{3.84in}|}{Redefinir senha} \\
\hhline{--}
%row no:3
\multicolumn{1}{|p{1.35in}}{\textbf{Ator}} & 
\multicolumn{1}{|p{3.84in}|}{Separador e Catador} \\
\hhline{--}
%row no:4
\multicolumn{1}{|p{1.35in}}{\textbf{Pré-condições}} & 
\multicolumn{1}{|p{3.84in}|}{O usuário deve estar cadastrado no sistema.} \\
\hhline{--}
%row no:5
\multicolumn{1}{|p{1.35in}}{\textbf{Pós-condições}} & 
\multicolumn{1}{|p{3.84in}|}{A senha será recuperada após o preenchimento do \textit{e-mail}.} \\
\hhline{--}
%row no:6
\multicolumn{1}{|p{1.35in}}{\textbf{Cenário Principal}} & 
\multicolumn{1}{|p{3.84in}|}{\begin{enumerate}[label*={\fontsize{12pt}{12pt}\selectfont \arabic*.}]
	\item O usuário solicita a recuperação da senha; \par 	\item Em seguida, informa o \textit{e-mail}; \par 	\item Um \textit{e-mail }para a redefinição da senha é enviado para o usuário. 
\end{enumerate}} \\
\hhline{--}

\end{tabular}
\end{quadro}

%%%%%%%%%%%%%%%%%%%% Table No: 7 ends here %%%%%%%%%%%%%%%%%%%%

O caso de uso Recuperar Senha (\autoref{quad:recusenha}), citado acima é responsável por avaliar as ações necessárias para realizar a redefinição da senha esquecida pelo usuário.

O \autoref{quad:duvida} descreve o passo a passo para tirar dúvidas.

%%%%%%%%%%%%%%%%%%%% Table No: 8 starts here %%%%%%%%%%%%%%%%%%%%


\begin{quadro}[H]
\caption{Tirar Dúvidas}
\label{quad:duvida}
\centering
\begin{tabular}{p{1.35in}p{4.33in}}
\hline
%row no:1
\multicolumn{1}{|p{1.35in}}{\textbf{Caso de Uso}} & 
\multicolumn{1}{|p{4.33in}|}{Tirar dúvidas} \\
\hhline{--}
%row no:2
\multicolumn{1}{|p{1.35in}}{\textbf{Descrição}} & 
\multicolumn{1}{|p{4.33in}|}{Explicar a definição dos perfis.} \\
\hhline{--}
%row no:3
\multicolumn{1}{|p{1.35in}}{\textbf{Ator}} & 
\multicolumn{1}{|p{4.33in}|}{Separador e Catador} \\
\hhline{--}
%row no:4
\multicolumn{1}{|p{1.35in}}{\textbf{Pré-condições}} & 
\multicolumn{1}{|p{4.33in}|}{O usuário não precisa ser cadastrado ou estar logado no sistema.} \\
\hhline{--}
%row no:5
\multicolumn{1}{|p{1.35in}}{\textbf{Pós-condições}} & 
\multicolumn{1}{|p{4.33in}|}{O sistema deve ser iniciado.} \\
\hhline{--}
%row no:6
\multicolumn{1}{|p{1.35in}}{\textbf{Cenário Principal}} & 
\multicolumn{1}{|p{4.33in}|}{\begin{enumerate}[label*={\fontsize{12pt}{12pt}\selectfont \arabic*.}]
	\item O usuário clica no ícone de dúvidas; \par 	\item O sistema redireciona o usuário para a tela informativa.
\end{enumerate}} \\
\hhline{--}

\end{tabular}
\end{quadro}

%%%%%%%%%%%%%%%%%%%% Table No: 8 ends here %%%%%%%%%%%%%%%%%%%%

O caso de uso Tirar Dúvidas (\autoref{quad:duvida}), citado acima é responsável por avaliar as ações necessárias para esclarecer a definição dos perfis. 

O \autoref{quad:logoff} descreve o passo a passo para realizar \textit{logoff}.

%%%%%%%%%%%%%%%%%%%% Table No: 9 starts here %%%%%%%%%%%%%%%%%%%%

\begin{quadro}[H]
\caption{Realizar Logoff}
\label{quad:logoff}
\centering
\begin{tabular}{p{1.35in}p{3.15in}}
\hline
%row no:1
\multicolumn{1}{|p{1.35in}}{\textbf{Caso de Uso}} & 
\multicolumn{1}{|p{3.15in}|}{Realizar logoff} \\
\hhline{--}
%row no:2
\multicolumn{1}{|p{1.35in}}{\textbf{Descrição}} & 
\multicolumn{1}{|p{3.15in}|}{Sair do aplicativo} \\
\hhline{--}
%row no:3
\multicolumn{1}{|p{1.35in}}{\textbf{Ator}} & 
\multicolumn{1}{|p{3.15in}|}{Separador e Catador} \\
\hhline{--}
%row no:4
\multicolumn{1}{|p{1.35in}}{\textbf{Pré-condições}} & 
\multicolumn{1}{|p{3.15in}|}{O usuário precisa estar cadastrado no sistema.} \\
\hhline{--}
%row no:5
\multicolumn{1}{|p{1.35in}}{\textbf{Pós-condições}} & 
\multicolumn{1}{|p{3.15in}|}{O usuário é deslogado do sistema.} \\
\hhline{--}
%row no:6
\multicolumn{1}{|p{1.35in}}{\textbf{Cenário Principal}} & 
\multicolumn{1}{|p{3.15in}|}{\begin{enumerate}[label*={\fontsize{12pt}{12pt}\selectfont \arabic*.}]
	\item O usuário solicita a saída da aplicação; \par 	\item O sistema desloga o usuário.
\end{enumerate}} \\
\hhline{--}

\end{tabular}
\end{quadro}

%%%%%%%%%%%%%%%%%%%% Table No: 9 ends here %%%%%%%%%%%%%%%%%%%%

O caso de uso Realizar \textit{Logoff} (\autoref{quad:logoff}), citado acima é responsável por avaliar as ações necessárias para o término do uso do sistema. 

O \autoref{quad:local} descreve o passo a passo para buscar a localização.

%%%%%%%%%%%%%%%%%%%% Table No: 10 starts here %%%%%%%%%%%%%%%%%%%%

\begin{quadro}[H]
\caption{Informar Localização}
\label{quad:local}
\centering
\begin{tabular}{p{1.35in}p{3.94in}}
\hline
%row no:1
\multicolumn{1}{|p{1.35in}}{\textbf{Caso de Uso}} & 
\multicolumn{1}{|p{3.94in}|}{Informar localização} \\
\hhline{--}
%row no:2
\multicolumn{1}{|p{1.35in}}{\textbf{Descrição}} & 
\multicolumn{1}{|p{3.94in}|}{Informar a localização do material descartado.} \\
\hhline{--}
%row no:3
\multicolumn{1}{|p{1.35in}}{\textbf{Ator}} & 
\multicolumn{1}{|p{3.94in}|}{Separador} \\
\hhline{--}
%row no:4
\multicolumn{1}{|p{1.35in}}{\textbf{Pré-condições}} & 
\multicolumn{1}{|p{3.94in}|}{O usuário deve estar cadastrado no sistema.} \\
\hhline{--}
%row no:5
\multicolumn{1}{|p{1.35in}}{\textbf{Pós-condições}} & 
\multicolumn{1}{|p{3.94in}|}{O sistema deve ser executado.} \\
\hhline{--}
%row no:6
\multicolumn{1}{|p{1.35in}}{\textbf{Cenário Principal}} & 
\multicolumn{1}{|p{3.94in}|}{\begin{enumerate}[label*={\fontsize{12pt}{12pt}\selectfont \arabic*.}]
	\item O usuário informa o local para a coleta do material.
\end{enumerate}} \\
\hhline{--}

\end{tabular}
\end{quadro}

%%%%%%%%%%%%%%%%%%%% Table No: 10 ends here %%%%%%%%%%%%%%%%%%%%

O caso de uso Informar Localização (\autoref{quad:local}), citado acima é responsável por avaliar as ações necessárias para a localização do material informada pelo usuário. 

O \autoref{quad:recolhe} descreve o passo a passo para solicitação do recolhimento do material descartado.

%%%%%%%%%%%%%%%%%%%% Table No: 11 starts here %%%%%%%%%%%%%%%%%%%%

\begin{quadro}[H]
\caption{Solicitar Recolhimento}
\label{quad:recolhe}
\centering
\begin{tabular}{p{1.35in}p{4.06in}}
\hline
%row no:1
\multicolumn{1}{|p{1.35in}}{\textbf{Caso de Uso}} & 
\multicolumn{1}{|p{4.06in}|}{Solicitar Recolhimento} \\
\hhline{--}
%row no:2
\multicolumn{1}{|p{1.35in}}{\textbf{Descrição}} & 
\multicolumn{1}{|p{4.06in}|}{O usuário requisita o recolhimento do material.} \\
\hhline{--}
%row no:3
\multicolumn{1}{|p{1.35in}}{\textbf{Ator}} & 
\multicolumn{1}{|p{4.06in}|}{Separador} \\
\hhline{--}
%row no:4
\multicolumn{1}{|p{1.35in}}{\textbf{Pré-condições}} & 
\multicolumn{1}{|p{4.06in}|}{O usuário precisa estar cadastrado.} \\
\hhline{--}
%row no:5
\multicolumn{1}{|p{1.35in}}{\textbf{Pós-condições}} & 
\multicolumn{1}{|p{4.06in}|}{O sistema deve ser iniciado.} \\
\hhline{--}
%row no:6
\multicolumn{1}{|p{1.35in}}{\textbf{Cenário Principal}} & 
\multicolumn{1}{|p{4.06in}|}{\begin{enumerate}[label*={\fontsize{12pt}{12pt}\selectfont \arabic*.}]
	\item O usuário informa o tipo de rejeito a ser descartado; \par 	\item Em seguida, informa a localização do material; \par 	\item Por fim, solicita a coleta do rejeito.
\end{enumerate}} \\
\hhline{--}

\end{tabular}
\end{quadro}

%%%%%%%%%%%%%%%%%%%% Table No: 11 ends here %%%%%%%%%%%%%%%%%%%%

O caso de uso Solicitar Recolhimento (\autoref{quad:recolhe}), citado acima é responsável por avaliar as ações necessárias para realizar a coleta do material descartado. 

O \autoref{quad:lemlogin} descreve o passo a passo para o salvamento dos dados de autenticação do usuário.

%%%%%%%%%%%%%%%%%%%% Table No: 12 starts here %%%%%%%%%%%%%%%%%%%%
\begin{quadro}
\caption{Lembrar Login}
\label{quad:lemlogin}
\centering
\begin{tabular}{p{1.35in}p{4.40in}}
\hline
%row no:1
\multicolumn{1}{|p{1.35in}}{\textbf{Caso de Uso}} & 
\multicolumn{1}{|p{4.40in}|}{Lembrar login} \\
\hhline{--}
%row no:2
\multicolumn{1}{|p{1.35in}}{\textbf{Descrição}} & 
\multicolumn{1}{|p{4.40in}|}{Permite que os dados de autenticação, fiquem salvos na tela inicial da aplicação.} \\
\hhline{--}
%row no:3
\multicolumn{1}{|p{1.35in}}{\textbf{Ator}} & 
\multicolumn{1}{|p{4.40in}|}{Catador e Separador} \\
\hhline{--}
%row no:4
\multicolumn{1}{|p{1.35in}}{\textbf{Cenário Principal}} & 
\multicolumn{1}{|p{4.40in}|}{\begin{enumerate}[label*={\fontsize{12pt}{12pt}\selectfont \arabic*.}]
	\item O usuário preenche os dados de autenticação; \par 	
	\item Selecionar o recurso $``$Lembrar Login$"$ ; \par 	
	\item O sistema armazena os dados no dispositivo.
\end{enumerate}} \\
\hhline{--}

\end{tabular}
\end{quadro}

%%%%%%%%%%%%%%%%%%%% Table No: 12 ends here %%%%%%%%%%%%%%%%%%%%

O caso de uso Lembrar Senha (\autoref{quad:lemlogin}), citado acima é responsável por avaliar as ações necessárias, para salvar os dados de autenticação do usuário. 

O \autoref{quad:mapa} descreve o passo a passo para a visualização do mapa.

%%%%%%%%%%%%%%%%%%%% Table No: 13 starts here %%%%%%%%%%%%%%%%%%%%
\begin{quadro}[H]
\caption{Visualizar Mapa}
\label{quad:mapa}
 			\centering
\begin{tabular}{p{1.28in}p{4.33in}}
\hline
%row no:1
\multicolumn{1}{|p{1.28in}}{\textbf{Caso de Uso}} & 
\multicolumn{1}{|p{4.33in}|}{Visualizar Mapa} \\
\hhline{--}
%row no:2
\multicolumn{1}{|p{1.28in}}{\textbf{Descrição}} & 
\multicolumn{1}{|p{4.33in}|}{Permite que o usuário visualize sua localização ou a rota a ser percorrida, conforme o tipo de perfil do usuário.} \\
\hhline{--}
%row no:3
\multicolumn{1}{|p{1.28in}}{\textbf{Ator}} & 
\multicolumn{1}{|p{4.33in}|}{Catador e Separador} \\
\hhline{--}
%row no:4
\multicolumn{1}{|p{1.28in}}{\textbf{Pré-condições}} & 
\multicolumn{1}{|p{4.33in}|}{O usuário deve estar cadastrado no sistema.} \\
\hhline{--}
%row no:5
\multicolumn{1}{|p{1.28in}}{\textbf{Cenário Principal}} & 
\multicolumn{1}{|p{4.33in}|}{\textbf{Usuário Catador} \par \begin{enumerate}
	\item O mapa será visualizado, quando o catador traçar a rota.
\end{enumerate} \par \textbf{Usuário Separador} \par \begin{enumerate}[label*={\fontsize{12pt}{12pt}\selectfont \arabic*.}]
	\item Sua localização será visualizada no mapa.
\end{enumerate}} \\
\hhline{--}

\end{tabular}
\end{quadro}

%%%%%%%%%%%%%%%%%%%% Table No: 13 ends here %%%%%%%%%%%%%%%%%%%%

O caso de uso Visualizar Mapa (\autoref{quad:mapa}), citado acima é responsável por avaliar as ações necessárias, para a visualização do mapa.

O \autoref{quad:material} descreve o passo a passo para escolha do tipo de material.

%%%%%%%%%%%%%%%%%%%% Table No: 14 starts here %%%%%%%%%%%%%%%%%%%%

\begin{quadro}[H]
\caption{Escolher Tipo de Material}
\label{quad:material}
\centering
\begin{tabular}{p{1.35in}p{4.35in}}
\hline
%row no:1
\multicolumn{1}{|p{1.35in}}{\textbf{Caso de Uso}} & 
\multicolumn{1}{|p{4.35in}|}{Escolher Tipo de Material} \\
\hhline{--}
%row no:2
\multicolumn{1}{|p{1.35in}}{\textbf{Descrição}} & 
\multicolumn{1}{|p{4.35in}|}{Permite que o usuário selecione o tipo de material a ser descartado.} \\
\hhline{--}
%row no:3
\multicolumn{1}{|p{1.35in}}{\textbf{Ator}} & 
\multicolumn{1}{|p{4.35in}|}{Separador} \\
\hhline{--}
%row no:4
\multicolumn{1}{|p{1.35in}}{\textbf{Pré-condições}} & 
\multicolumn{1}{|p{4.35in}|}{O usuário deve estar cadastrado no sistema.} \\
\hhline{--}
%row no:5
\multicolumn{1}{|p{1.35in}}{\textbf{Cenário Principal}} & 
\multicolumn{1}{|p{4.35in}|}{\begin{enumerate}[label*={\fontsize{12pt}{12pt}\selectfont \arabic*.}]
	\item O sistema irá apresentar uma lista de materiais; \par 	\item O usuário informa qual tipo de material será descartado.
\end{enumerate}} \\
\hhline{--}

\end{tabular}
\end{quadro}

%%%%%%%%%%%%%%%%%%%% Table No: 14 ends here %%%%%%%%%%%%%%%%%%%%

O caso de uso Escolher Tipo de Material (\autoref{quad:material}), citado acima é responsável por avaliar as ações necessárias, para a escolha do tipo de material descartado.

O \autoref{quad:rota} descreve o passo a passo para traçar a rota.


%%%%%%%%%%%%%%%%%%%% Table No: 15 starts here %%%%%%%%%%%%%%%%%%%%


\begin{quadro}[H]
\caption{Traçar Rota}
\label{quad:rota}
\centering
\begin{tabular}{p{1.35in}p{4.30in}}
\hline
%row no:1
\multicolumn{1}{|p{1.35in}}{\textbf{Caso de Uso}} & 
\multicolumn{1}{|p{4.30in}|}{Traçar Rota} \\
\hhline{--}
%row no:2
\multicolumn{1}{|p{1.35in}}{\textbf{Descrição}} & 
\multicolumn{1}{|p{4.30in}|}{Permite que o usuário visualize através do mapa, a rota a ser percorrida.} \\
\hhline{--}
%row no:3
\multicolumn{1}{|p{1.35in}}{\textbf{Ator}} & 
\multicolumn{1}{|p{4.30in}|}{Catador } \\
\hhline{--}
%row no:4
\multicolumn{1}{|p{1.35in}}{\textbf{Pré-condições}} & 
\multicolumn{1}{|p{4.30in}|}{O usuário deve estar cadastrado no sistema} \\
\hhline{--}
%row no:5
\multicolumn{1}{|p{1.35in}}{\textbf{Cenário Principal}} & 
\multicolumn{1}{|p{4.30in}|}{\begin{enumerate}[label*={\fontsize{12pt}{12pt}\selectfont \arabic*.}]
	\item O usuário informa a origem e o destino; \par 	
	\item O sistema exibe a rota no mapa com a duração e a distância.
\end{enumerate}} \\
\hhline{--}

\end{tabular}
\end{quadro}

%%%%%%%%%%%%%%%%%%%% Table No: 15 ends here %%%%%%%%%%%%%%%%%%%%

O caso de uso Traçar Rota (\autoref{quad:rota}), citado acima é responsável por avaliar as ações necessárias, para traçar a rota no mapa e informar sua duração e distância.



% \chapter{Resultados e Discussões}
\label{cap:resultados}
Este capítulo apresenta os resultados obtidos com a execução dos testes de carga descritos no \autoref{cap:estudo_caso1}. Além disso, há uma discussão geral sobre o processo de desenvolvimento do caso de uso com a utilização da \acrfull{ams} e \acrfull{ddd}.

\section{Resultados}
Essa seção apresenta uma análise gráfica dos resultados obtidos com a execução dos testes de carga. Os gráficos foram gerados pelo \english{AWS CloudWatch} a partir das métricas coletadas dos serviços de computação e banco de dados da \english{Amazon Web Services} durante a execução dos testes. Os dados apresentados apresentam a média dos serviços.

\subsection{Utilização da CPU}
Na \autoref{fig:cpu-utilization} é possível observar a utilização da CPU durante a execução dos testes de carga. Durante toda a simulação, a utilização de CPU se manteve em torno de 35\%, um valor considerado baixo. Isso indica que o sistema é capaz de suportar um número maior de requisições sem que haja um aumento significativo na utilização de CPU. Esse valor também está abaixo do alvo de 70\% estabelecido anteriormente. 

\begin{figure}[H]
    \centering
    \caption{Utilização da CPU}
    \includegraphics[width=0.8\textwidth]{media/cpu-usage.png}
    \fonte{o autor}
    \label{fig:cpu-utilization}
\end{figure}

\subsection{Utilização da Memória RAM}
Na \autoref{fig:memory-utilization} é possível observar a utilização da memória RAM durante a execução dos testes de carga. Houve um uso muito baixo de memória, com uma média em torno de 7\%. Isso indica que o sistema como um todo é \english{CPU-bound}, termo utilizado para descrever sistemas que são limitados pela CPU e não pela memória. Essa métrica também ficou abaixo do alvo.

\begin{figure}[H]
    \centering
    \caption{Utilização da Memória RAM}
    \includegraphics[width=0.8\textwidth]{media/memory-usage.png}
    \fonte{o autor}
    \label{fig:memory-utilization}
\end{figure}

\subsection{Tempo de Resposta}
Na \autoref{fig:response-time} é possível observar o tempo de resposta das requisições durante a simulação. Utilizando como base o P90, o tempo de resposta se manteve em torno de 1 segundo, um valor considerado aceitável. Esse valor também está abaixo do alvo de 2 segundos estabelecido anteriormente.

\begin{figure}[H]
    \centering
    \caption{Tempo de Resposta}
    \includegraphics[width=0.8\textwidth]{media/response-time.png}
    \fonte{o autor}
    \label{fig:response-time}
\end{figure}

\subsection{Porcentagem de erros}
Na \autoref{fig:error-rate} é possível observar a porcentagem de erros durante a execução dos testes de carga. Durante toda a simulação, a porcentagem de erros se manteve em 0.2\% (média), indicando que o sistema é capaz de suportar um grande número de requisições sem impactar os usuários. Esse valor também está abaixo do alvo de 1\% estabelecido anteriormente.

\begin{figure}[H]
    \centering
    \caption{Porcentagem de Erros}
    \includegraphics[width=0.8\textwidth]{media/errors.png}
    \fonte{o autor}
    \label{fig:error-rate}
\end{figure}

\section{Trabalhos Futuros}
Como trabalhos futuros, é possível destacar a construção de novos estudos de caso com a utilização de \acrshort{ddd} e \acrshort{ams} em outros contextos complexos como, por exemplo, sistemas financeiros, sistemas de logísticas e sistemas de saúde. Além disso, um trabalho futuro interessante seria a comparação de entre dois sistemas de domínios iguais, um desenvolvido com as técnicas apresentadas nesse trabalho e um outro somente com \acrshort{ddd} ou \acrshort{ams}. Assim, seria possível quantificar os benefícios e desvantagens da utilização dessas tecnologias em conjunto.

\section{Discussões}
Essa seção apresenta uma discussão geral sobre o processo de desenvolvimento do caso de uso com a utilização da \acrfull{ams} e \acrfull{ddd}. A utilização dessas tecnologias trouxe diversos desafios e benefícios ao desenvolvimento do sistema.

A separação de serviços em microsserviços permitiu que cada parte do sistema fosse desenvolvida de forma independente, facilitando a manutenção e evolução do sistema. Além disso, a utilização de \acrfull{ddd} permitiu que o domínio do negócio fosse modelado de forma mais clara e eficiente, facilitando o entendimento do sistema como um todo.

Da mesma forma, como cada microsserviço tem um foco específico, se torna mais fácil o entendimento de como cada parte do sistema funciona. Isso também facilita a escalabilidade do sistema, uma vez que é possível escalar apenas os serviços que estão sobrecarregados.

A utilização de \acrfull{ams} e \acrfull{ddd} trouxe diversos desafios ao desenvolvimento do sistema. O maior deles foi a necessidade de um maior esforço de \english{design upfront} para garantir a correta separação de serviços e a definição do domínio do negócio. Isso se deve ao fato de que a separação de serviços em microsserviços e a utilização de \acrfull{ddd} requerem um maior entendimento do negócio e da arquitetura do sistema.

Além disso, a complexidade aumentada para criação e execução de testes também foi um desafio. Como cada microsserviço é um sistema independente, é necessário criar testes para cada parte do sistema, o que aumenta a complexidade dos testes.

Outro desafio foi o maior custo computacional para executar o sistema localmente. Como cada microsserviço é uma aplicação independente, possui seu próprio processo em nível de sistema operacional, o que aumenta o custo computacional para executar o sistema localmente. Assim, os desenvolvedores precisam de máquinas mais potentes.

Por fim, a maior complexidade de deploy também foi um desafio. É necessário criar mais serviços, configurar máquinas virtuais, \english{load balancers} e banco de dados. Além disso, é importante configurar corretamente a rede para garantir a comunicação entre os serviços. Inicialmente, se tem um custo maior para manter o sistema em produção. Porém, com o crescimento do tráfego, a escalabilidade do sistema se torna mais fácil.

% \chapter{Conclusão}
\label{cap:conclusão}

Este trabalho apresentou um estudo de caso sobre a utilização de \acrfull{ddd} e \acrfull{ams} no desenvolvimento de um sistema para uma locadora de veículos. O objetivo foi avaliar como essas tecnologias podem ser utilizadas para aumentar a escalabilidade e resiliência de sistemas complexos. Com base dos resultados obtidos com a execução dos testes de carga, é possível concluir que essa abordagem é eficaz para a construção de sistemas distribuídos. Além disso, nota-se como diferentes partes do sistema são expostos a tráfegos distintos, o que permite a escalabilidade independente de cada serviço.

Com a modelagem do sistema através de \acrshort{ddd}, foi possível definir limites claros entre os diferentes contextos do negócio com a utilização de \acrfull{bc}. Além disso, a utilização de \acrshort{ddd} permitiu a definição de um modelo de domínio rico e expressivo, que reflete de forma fiel as regras de negócio da locadora de veículos.

Por otro lado, o desenvolvimento deste estudo de caso trouxe diversos desafios como a necessidade de um maior esforço de \english{design upfront} para garantir a correta separação de serviços e a definição do domínio do negócio. Além disso, percebe-se uma maior complexidade para realização de testes e depuração de problemas, uma vez que o sistema é composto por diversos serviços independentes.

Este trabalho cumpriu com os objetivos propostos, na medida que apresentou uma estratégia para transformar requisitos funcionais em um \english{design} com \acrshort{ams} e \acrshort{ddd}, prover informações relevantes para definição dos estilo de comunicação adequado entre microsserviços e demonstrar desempenho e escalabilidade do sistema desenvolvido através de testes de carga.


\postextual

\bibliography{Referencias}

% Imprime uma página indicando o início dos apêndices
% \partapendices
%\part*{Apêndices}

%\begin{apendicesenv}

%
\chapter{Questionário de Avaliação do Protótipo}

\begin{figure}[H]
	\begin{Center}
		\includegraphics[width=6.00in]{media/questionariopaint.png}
		\label{fig:questionario}
	\end{Center}
\end{figure}

\chapter{Códigos fonte comentados}

\section{Salvar usuário catador no banco de dados}

O \autoref{cod:cadbanc} é responsável por salvar o usuário catador no \textit{firebase realtime database}. A linha 3 do \autoref{cod:cadbanc}, consiste na criação do nó (\textit{child}) denominado “Catadores”, onde cada catador possui um identificador (\textit{id}), onde seus dados são salvos (\textit{setValue}).

\begin{codigo}[H]
\begin{lstlisting}[language=Java]
public void salvar(){
    DatabaseReference databaseReference = ConfiguracaoFirebase.getFirebase();
    databaseReference.child("Catadores").child(getId()).setValue(this);
}
\end{lstlisting}
\caption{Salvar usuário catador no banco de dados}
\label{cod:cadbanc}
\end{codigo}


\section{Salvar usuário separador no banco de dados}

\begin{codigo}[H]
	\begin{lstlisting}[language=Java]
public void salvar(){
    DatabaseReference databaseReference = ConfiguracaoFirebase.getFirebase();
    databaseReference.child("Separadores").child(getId()).setValue(this);
}
   	\end{lstlisting}
   	\caption{Salvar usuário separador no banco de dados}
   	\label{cod:cadsep}
\end{codigo}

O \autoref{cod:cadsep} é responsável por salvar o usuário separador no \textit{firebase realtime database}. A linha 3 do \autoref{cod:cadbanc}, consiste na criação do nó (\textit{child}) denominado “Separadores”, onde cada separador possui um identificador (\textit{id}), onde seus dados são salvos (\textit{setValue}).


\section{Cadastrar usuário catador}

\begin{codigo}[H]
	\begin{lstlisting}[language=Java]
private void cadastrarCatador(){
    firebaseAuth = ConfiguracaoFirebase.getFirebaseAuth();
    firebaseAuth.createUserWithEmailAndPassword(
            catador.getEmail(),
            catador.getSenha()
    ).addOnCompleteListener(CadastroCatador.this, new OnCompleteListener<AuthResult>() {
        @Override
        public void onComplete(@NonNull Task<AuthResult> task) {
            if(task.isSuccessful()){
                Toast.makeText(CadastroCatador.this,"Sucesso ao cadastrar catador",Toast.LENGTH_LONG).show();
                FirebaseUser firebaseUser = task.getResult().getUser();
                catador.setId(firebaseUser.getUid());
                catador.salvar();
                startActivity(new Intent(CadastroCatador.this, Login.class));
       	\end{lstlisting}
       	\caption{Cadastrar usuário catador na aplicação}
       	\label{cod:cat}
\end{codigo}

O \autoref{cod:cat} consiste no cadastro do usuário catador. A função \textit{createUserWithEmailAndPassword}, que se encontra na linha 3, é responsável pela criação da autenticação dos usuário através do \textit{e-mail} e a senha. Em caso de sucesso, o sistema irá exibir a seguinte mensagem “Sucesso ao cadastrar catador” e o usuário será redirecionado para a tela de \textit{login}.

\section{Cadastrar usuário separador}

\begin{codigo}[H]
	\begin{lstlisting}[language=Java]
private void cadastrarSeparador(){
    firebaseAuth = ConfiguracaoFirebase.getFirebaseAuth();
    firebaseAuth.createUserWithEmailAndPassword(
            separador.getEmail(),
            separador.getSenha()
    ).addOnCompleteListener(CadastroSeparador.this, new OnCompleteListener<AuthResult>() {
        @Override
        public void onComplete(@NonNull Task<AuthResult> task) {
            if(task.isSuccessful()){
                Toast.makeText(CadastroSeparador.this,"Sucesso ao cadastrar separador",Toast.LENGTH_LONG).show();
                FirebaseUser firebaseUser = task.getResult().getUser();
                separador.setId(firebaseUser.getUid());
                separador.salvar();
                startActivity(new Intent(CadastroSeparador.this, Login.class));
       	\end{lstlisting}
       	\caption{Cadastrar usuário separador}
       	\label{cod:sep}
\end{codigo}

O \autoref{cod:sep} consiste no cadastro do usuário separador. A função \textit{createUserWithEmailAndPassword}, que se encontra na linha 3, é responsável pela criação da autenticação do usuário através do \textit{e-mail} e a senha. Em caso de sucesso, o sistema irá exibir a seguinte mensagem “Sucesso ao cadastrar separador” e o usuário será redirecionado para a tela de \textit{login}.

\section{Redefinir a senha}

\begin{codigo}[H]
	\begin{lstlisting}[language=Java]
	private void redefinirsenha(){
        firebaseAuth = FirebaseAuth.getInstance();
        firebaseAuth.sendPasswordResetEmail(recuperar.getEmail())
                .addOnCompleteListener(new OnCompleteListener<Void>() {
                    @Override
                    public void onComplete(@NonNull Task<Void> task) {
                        if(task.isSuccessful()) {
                            Toast.makeText(RecuperarSenha.this, "E-mail enviado.", Toast.LENGTH_LONG).show();
                            startActivity(new Intent(RecuperarSenha.this, Login.class));
                        }else{
                            Toast.makeText(RecuperarSenha.this,"E-mail nao cadastrado.",Toast.LENGTH_LONG).show();
                        }
                    }
                });
    }
    \caption{Redefinir a senha}
    \label{cod:senha}
 	\end{lstlisting}
\end{codigo}

O \autoref{cod:senha} consiste na redefinição da senha do usuário. A linha 3 verifica se o \textit{e-mail} informado pelo usuário está cadastrado no \textit{firebase auth}. Em caso de sucesso, o sistema irá enviar um \textit{e-mail} para a redefinição da senha e o usuário será redirecionado para a tela de \textit{login}. Caso contrário o sistema irá exibir a seguinte mensagem “\textit{E-mail} não cadastrado”.

\section{Deslogar usuário}
\begin{codigo}[H]
	\begin{lstlisting}[language=Java]
private void deslogarUsuario() {
        firebaseAuth.signOut();
        Toast.makeText(MapsActivity.this, "Usuario deslogado!", Toast.LENGTH_LONG).show();
        Intent intent = new Intent(MapsActivity.this, Login.class);
        startActivity(intent);
        finish();
    \end{lstlisting}
    \caption{Deslogar usuário}
    \label{cod:des}
\end{codigo}

O \autoref{cod:des} consiste no \textit{logout} do usuário. A função \textit{signOut} que encontra na linha 2 é responsável por desconectar o usuário da aplicação. A linha 3 tem a função de exibir para o usuário a seguinte mensagem "Usuario deslogado". A linha 4 tem o papel de redirecionar o usuário para a tela de \textit{login}.

\section{Validar separador}

\begin{codigo}[H]
	\begin{lstlisting}[language=Java]
private void validarSeparador() {
    firebaseAuth2 = ConfiguracaoFirebase.getFirebaseAuth();
    firebaseAuth2.signInWithEmailAndPassword(
        separador.getEmail(),
        separador.getSenha()
    ).addOnCompleteListener(new OnCompleteListener<AuthResult>() {
        @Override
        public void onComplete(@NonNull Task<AuthResult> task) {
            if (task.isSuccessful()) {
                FirebaseUser currentUser = FirebaseAuth.getInstance().getCurrentUser();
                assert currentUser != null;
                String id = currentUser.getUid();
                DatabaseReference j = FirebaseDatabase.getInstance().getReference().child("Separadores").child(id);
                j.addValueEventListener(new ValueEventListener() {
                    @Override
                    public void onDataChange(@NonNull DataSnapshot dataSnapshot)
                    {
                        String userType = (String) dataSnapshot.child("userType").getValue();
                        if (userType != null && userType.equals("Separadores")) {
                            Toast.makeText(Login.this, "Login efetuado com sucesso!", Toast.LENGTH_SHORT).show();
                            Intent intentResident = new Intent(Login.this, Materiais.class);
                            startActivity(intentResident);
                            finish();
                        }
                }
    	\end{lstlisting}
    	\caption{Validar separador}
    	\label{cod:valsep}
\end{codigo}

O \autoref{cod:valsep} consiste na validação do usuário separador. A verificação do tipo de usuário separador é realizada através da referência do nó "Separador", do identificador e do tipo de usuário.


\section{Validar catador}

\begin{codigo}[H]
	\begin{lstlisting}[language=Java]
private void validarCatador() {
    firebaseAuth2 = ConfiguracaoFirebase.getFirebaseAuth();
    firebaseAuth2.signInWithEmailAndPassword(
        catador.getEmail(),
        catador.getSenha()
    ).addOnCompleteListener(new OnCompleteListener<AuthResult>() {
        @Override
        public void onComplete(@NonNull Task<AuthResult> task) {
            if (task.isSuccessful()) {
                FirebaseUser currentUser = FirebaseAuth.getInstance().getCurrentUser();
                assert currentUser != null;
                String id = currentUser.getUid();
                DatabaseReference j = FirebaseDatabase.getInstance().getReference().child("Catadores").child(id);
                j.addValueEventListener(new ValueEventListener() {
                    @Override
                    public void onDataChange(@NonNull DataSnapshot dataSnapshot)
                    {
                        String userType = (String) dataSnapshot.child("userType").getValue();
                        if (userType != null && userType.equals("Catador")) {
                            Toast.makeText(Login.this, "Login efetuado com sucesso!", Toast.LENGTH_SHORT).show();
                            Intent intentResident = new Intent(Login.this, Servicos.class);
                            startActivity(intentResident);
                            finish();
                        }
                }
    	\end{lstlisting}
    	\caption{Validar catador}
    	\label{cod:valcat}
\end{codigo}

O \autoref{cod:valcat} consiste na validação do usuário catador. A verificação do tipo de usuário catador é realizada através da referência do nó "Catadores", do identificador e do tipo de usuário.

\section{Traçar rota}

\begin{codigo}[H]
	\begin{lstlisting}[language=Java]
	private void parseJSon(String data) throws JSONException {
        if (data == null)
            return;
        List<Route> routes = new ArrayList<Route>();
        JSONObject jsonData = new JSONObject(data);
        JSONArray jsonRoutes = jsonData.getJSONArray("routes");
        for (int i = 0; i < jsonRoutes.length(); i++) {
            JSONObject jsonRoute = jsonRoutes.getJSONObject(i);
            Route route = new Route();
            JSONObject overview_polylineJson = jsonRoute.getJSONObject("overview_polyline");
            JSONArray jsonLegs = jsonRoute.getJSONArray("legs");
            JSONObject jsonLeg = jsonLegs.getJSONObject(0);
            JSONObject jsonDistance = jsonLeg.getJSONObject("distance");
            JSONObject jsonDuration = jsonLeg.getJSONObject("duration");
            JSONObject jsonEndLocation = jsonLeg.getJSONObject("end_location");
            JSONObject jsonStartLocation = jsonLeg.getJSONObject("start_location");
            route.distance = new Distance(jsonDistance.getString("text"), jsonDistance.getInt("value"));
            route.duration = new Duration(jsonDuration.getString("text"), jsonDuration.getInt("value"));
            route.endAddress = jsonLeg.getString("end_address");
            route.startAddress = jsonLeg.getString("start_address");
            route.startLocation = new LatLng(jsonStartLocation.getDouble("lat"), jsonStartLocation.getDouble("lng"));
            route.endLocation = new LatLng(jsonEndLocation.getDouble("lat"), jsonEndLocation.getDouble("lng"));
            route.points = decodePolyLine(overview_polylineJson.getString("points"));
            routes.add(route);
        }
        listener.onDirectionFinderSuccess(routes);
    }
 	\end{lstlisting}
 	\caption{Traçar rota}
 	\label{cod:rot}
\end{codigo}

O \autoref{cod:rot} é responsável por traçar a rota informando sua distância e duração. 








%\end{apendicesenv}

%\begin{anexosenv}

%\input{pos/pos-anexos}

%\end{anexosenv}

\end{document}
